\begin{aosachapter}{The Dynamic Language Runtime and the Iron Languages}{s:ironlang}{Jeff Hardy}

The Iron languages are an informal group of language implementations with
``Iron'' in their names, in honour of the first one, IronPython. All of these
languages have at least one thing in common---they are dynamic languages that
target
the Common Language Runtime (CLR), which is more commonly known as the .NET
Framework\footnote{``CLR'' is the generic term; the .NET Framework is
Microsoft's implementation, and there is also the open-source Mono
implementation.}, and they are built on top of the Dynamic Language Runtime
(DLR).  The DLR is a set of libraries for the CLR that provide much better
support for dynamic languages on the CLR. IronPython and IronRuby are both used
in a few dozen closed and open source projects, and are both under active
development; the DLR, which started as an open-source project, is included as
part of the .NET Framework and Mono.

Architecturally, IronPython, IronRuby, and the DLR are both simple and
devilishly complex. From a high level, the designs are similar to many other
language implementations, with parsers and compilers and code generators;
however, look a little closer and the interesting details begin to
emerge: call sites, binders, adaptive compilation, and other techniques are
used to make dynamic languages perform nearly as fast as static languages on a
platform that was designed for static languages. 

\begin{aosasect1}{History}

The history of the Iron languages begins in 2003. Jim Hugunin had already
written an implementation of Python, called Jython, for the Java Virtual
Machine (JVM). At the time, the then-new .NET Framework Common Language Runtime
(CLR) was considered by some (exactly who, I'm not sure) to be poorly suited
for implementing dynamic languages such as Python. Having already implemented
Python on the JVM, Jim was curious as to how Microsoft could have made .NET so
much worse than Java. In a September 2006 blog post\footnote{\url{http://blogs.msdn.com/b/hugunin/archive/2006/09/05/741605.aspx}}, he wrote:

\begin{quotation}
I wanted to understand how Microsoft could have screwed up so badly that the
CLR was a worse platform for dynamic languages than the JVM.  My plan was to
take a couple of weeks to build a prototype implementation of Python on the CLR
and then to use that work to write a short pithy article called, ``Why the CLR
is a terrible platform for dynamic languages''.  My plans quickly changed as I
worked on the prototype, because I found that Python could run extremely well
on the CLR---in many cases noticeably faster than the C-based implementation.
For the standard
pystone\footnote{\url{http://ironpython.codeplex.com/wikipage?title=IP26RC1VsCPy26Perf}}
benchmark, IronPython on the CLR was about 1.7x faster than the C-based
implementation.
\end{quotation}

\noindent
(The ``Iron'' part of the name was a play on the name of Jim's company at the
time, Want of a Nail Software.)

Shortly afterwards, Jim was hired by Microsoft to make .NET an even better
platform for dynamic languages. Jim (and several others) developed the DLR by
factoring the language-neutral parts out of the original IronPython code. The
DLR was designed to provide a common core for implementing dynamic languages
for .NET, and was a major new feature of .NET 4.

At the same time as the DLR was announced (April 2007), Microsoft also
announced that, in addition to a new version of IronPython built on top of the
DLR (IronPython 2.0), they would be developing IronRuby on top of the DLR
to demonstrate the DLR's adaptability to multiple languages\footnote{In October
of 2010, Microsoft stopped developing IronPython and IronRuby and they became
independent open-source projects.}. Integration with dynamic languages using
the DLR would also be a major part of C\# and Visual Basic, with a new keyword
(\code{dynamic}) that allowed those languages to easily call into any language
implemented on the DLR, or any other dynamic data source. The CLR was already a
good platform for implementing static languages, and the DLR makes dynamic
languages a first-class citizen.

Other language implementations from outside of Microsoft also use the DLR,
including IronScheme\footnote{\url{http://ironscheme.codeplex.com/}} and
IronJS\footnote{\url{https://github.com/fholm/IronJS/}}. In addition,
Microsoft's PowerShell v3 will use the DLR instead of its own dynamic object
system.

\end{aosasect1}

\begin{aosasect1}{Dynamic Language Runtime Principles}

The CLR is designed with statically-typed languages in mind; the knowledge of
types is baked very deeply into the runtime, and one of its key assumptions is
that those types do not change---that a variable never changes its type, or
that a type never has any fields or members added or removed while the program
is running. This is fine for languages like C\# or Java, but dynamic languages,
by definition, do not follow those rules. The CLR also provides a common object
system for static types, which means that any .NET language can call objects
written in any other .NET language with no extra effort.

Without the DLR, every dynamic language would have to provide its own object
model; the various dynamic languages would not be able to call objects in
another dynamic language, and C\# would not be able to treat IronPython and
IronRuby equally. Thus, the heart of the DLR is a standard way of implementing
\emph{dynamic objects} while still allowing an object's behaviour to be
customized for a particular language by using \emph{binders}. It also includes
a mechanism known as \emph{call-site caching} for ensuring that dynamic
operations are as fast as possible, and a set of classes for building
\emph{expression trees}, which allow code to be stored as data and easily
manipulated.

The CLR also provides several other features that are useful to dynamic
languages, including a sophisticated garbage collector; a Just-in-Time (JIT)
compiler that converts Common Intermediate Language (IL) bytecode, which is
what .NET compilers output, into machine code at runtime; a runtime
introspection system, which allows dynamic languages to call objects written in
any static language; and finally, dynamic methods (also known as lightweight
code generation) that allow code to be generated at runtime and then executed
with only sightly more overhead than a static method call\footnote{The JVM
acquired a similar mechanism with \code{invokedynamic} in Java 7.}.

The result of the DLR design is that languages like IronPython and IronRuby can
call each other's objects (and those of any other DLR language), because they
have a common dynamic object model. Support for this object model was also
added to C\# 4 (with the \code{dynamic} keyword) and Visual~Basic~10 (in
addition to VB's existing method of ``late binding'') so that they can perform
dynamic calls on objects as well. The DLR thus makes dynamic languages
first-class citizens on .NET.

Interestingly, the DLR is entirely implemented as a set of libraries and can be
built and run on .NET 2.0 as well. No changes to the CLR are required to
implement it.

\end{aosasect1}

\begin{aosasect1}{Language Implementation Details}

Every language implementation has two basic stages---\emph{parsing} (the
front end) and \emph{code generation} (the backend). In the DLR, each language
implements its own front end, which contains the language parser and syntax tree
generator; the DLR provides a common backend that takes expression trees to
produce Intermediate Language (IL) for the CLR to consume; the CLR will pass
the IL to a Just-In-Time (JIT) compiler, which produces machine code to run on
the processor. Code that is defined at runtime (and run using \code{eval}) is
handled similarly, except that everything happens at the \code{eval} call site
instead of when the file is loaded.

There are a few different way to implement the key pieces of a language
front end, and while IronPython and IronRuby are very similar (they were
developed side-by-side, after all) they differ in a few key areas. Both
IronPython and IronRuby have fairly standard parser designs---both use a
\emph{tokenizer} (also known as a \emph{lexer}) to split the text into tokens,
and then the \emph{parser} turns those tokens into an \emph{abstract syntax
tree} (AST) that represents the program. However, the languages have completely
different implementations of these pieces.

\end{aosasect1}

\begin{aosasect1}{Parsing}

IronPython's tokenizer is in the \code{IronPython.Compiler.Tokenizer} class and
the parser is in the \code{IronPython.Compiler.Parser} class. The tokenizer is
a hand-written state machine that recognizes Python keywords, operators, and
names and produces the corresponding tokens. Each token also carries with it
any additional information (such as the value of a constant or name), as well
as where in the source the token was found, to aid in debugging. The parser then
takes this set of tokens and compares them to the Python grammar to see if it
matches legal Python constructs.

IronPython's parser is an LL(\kern-.05em1\kern-.05em) \emph{recursive descent parser}. The parser
will look at the incoming token, call a function if the token is allowed and
return an error if it is not. A recursive descent parser is built from a set of
mutually recursive functions; these functions ultimately implement a state
machine, with each new token triggering a state transition. Like the tokenizer,
IronPython's parser is written by hand.

IronRuby, on the other hand, has a tokenizer and parser generated by the
Gardens Point Parser Generator (GPPG). The parser is is described in the
\code{Parser\kern-.09em.\kern-.05emy}
file\footnote{\code{Languages/Ruby/Ruby/Compiler/Parser/Parser\kern-.09em.\kern-.05emy}}, which is a
\code{yacc}-format file that describes the grammar of IronRuby at a high level
using \emph{rules} that describe the grammar. GPPG then takes \code{Parser\kern-.09em.\kern-.05emy}
and creates the actual parser functions and tables; the result is a
\emph{table-based} LALR(1) parser. The generated tables are long arrays of
integers, where each integer represents a state; based on the current state and
the current token, the tables determine which state should be transitioned to
next. While IronPython's recursive descent parser is quite easy to read,
IronRuby's generated parser is not. The transition table is enormous (540
distinct states and over 45,000 transitions) and it is next to impossible to
modify it by hand.

Ultimately, this is an engineering tradeoff---IronPython's parser is simple
enough to modify by hand, but complex enough that it obscures the structure of
the language. The IronRuby parser, on the other hand, makes it much easier to
understand the structure of the language in the \code{Parser\kern-.09em.\kern-.05emy} file, but it is
now dependent on a third-party tool that uses a custom (albeit well-known)
domain-specific language and may have its own bugs or quirks. In this case, the
IronPython team didn't want to commit to a dependency on an external tool, 
while the
IronRuby team didn't mind.

What is clear, however, is how important state machines are to parsing, at
every phase. For any parsing task, no matter how simple, a state machine is
always the right answer.

The output of the parser for either language is an abstract syntax tree (AST).
This describes the structure of the program at a high level, with each node
mapping directly to a language construct---a statement or expression.  These
trees can be manipulated at runtime, often to make optimizations to the program
before compilation. However, a language's AST is tied to the language; the DLR
needs to operate on trees that do not contain any language-specific constructs,
only general ones.

\end{aosasect1}

\begin{aosasect1}{Expression Trees}

%% FIXME: DANGER - manual linebreaks ahead - make sure this is still working
%% in the final draft. --ARB
An \emph{expression tree} is also a representation of a program that can be
manipulated at \linebreak 
runtime, but in a lower-level, language-independent form. In
.NET, the node types are in the \linebreak 
\code{System.Linq.Expressions}
namespace\footnote{The namespace is a historical artifact; expression trees
were originally added in .NET 3.5 to implement LINQ---Language Integrated
Query---and the DLR expression trees extended that.}, and all of the node types
are derived from the abstract \code{Expression} class. These expression trees
cover more than just expressions, however, as there are node types for
\code{if} statements, \code{try} blocks, and loops as well; in some languages
(Ruby, for one) these are expressions and not statements.

There are nodes to cover almost every feature a programming language could
want. However, they tend to be defined at a fairly low level---instead of
having \code{ForExpression}, \code{WhileExpression}, etc., there is a single
\code{LoopExpression} which, when combined with a \code{GotoExpression}, can
describe any type of loop. To describe a language at a higher level, languages
can define their own node types by deriving from \code{Expression} and
overriding the \code{Reduce()} method, which returns another expression tree.
In IronPython, the parse tree is also a DLR expression tree, but it contains
many custom nodes that the DLR would not normally understand (such as
\code{ForStatement}). These custom nodes can be reduced to expression trees
that the DLR does understand (such as a combination of \code{LoopExpression}s
and \code{GotoExpression}s). A custom expression node can reduce to other
custom expression nodes, so the reduction proceeds recursively until only the
intrinsic DLR nodes remain. One key difference between IronPython and IronRuby
is that while IronPython's AST is also an expression tree, IronRuby's is not.
Instead, IronRuby's AST is transformed into an expression tree before moving
onto the next stage. It's arguable whether having the AST also be an expression
tree is actually useful, so IronRuby did not implement it that way.

Each node type knows how to reduce itself, and it can usually only be reduced
in one way. For transformations that come from code outside the 
tree---optimizations such as constant folding, for example, or IronPython's
implementation of Python generators---a subclass of the
\code{ExpressionVisitor} class is used. \code{ExpressionVisitor} has a
\code{Visit()} method that calls the \code{Accept()} method on
\code{Expression}, and subclasses of \code{Expression} override \code{Accept()}
to call a specific \code{Visit()} method on \code{ExpressionVisitor}, such as
\code{VisitBinary()}. This is a textbook implementation of the \emph{Visitor
pattern} from Gamma et al.---there's a fixed set of node types to visit, and an
infinite number of operations that could be performed upon them. When the
expression visitor visits a node, it usually recursively visits its children as
well, and its children, and so on down the tree. However, an
\code{ExpressionVisitor} can't actually modify the expression tree it is
visiting, because expression trees are immutable. If the expression visitor
needs to modify a node (such as removing children), it must produce a new node
that replaces the old one instead, and all of its parents as well.

Once an expression tree has been created, reduced, and visited, it ultimately
needs to be executed. While expression trees can be compiled directly to IL
code, IronPython and IronRuby pass them to an interpreter first, because
compiling directly to IL is expensive for code that may only be executed a
handful of times.

\end{aosasect1}

\begin{aosasect1}{Interpreting and Compilation}

One of the downsides to using a JIT compiler, like .NET does, is that it imposes
a time penalty when starting up because it takes time to convert the IL
bytecode into machine code that the processor can run. JIT compilation makes
the code much faster while running than using an interpreter, but the initial
cost can be prohibitive, depending on what is being done. For example, a
long-lived server process such as a web application will benefit from the JIT
because the startup time is mostly irrelevant but the per-request time is
critical, and it tends to run the same code repeatedly. On the other hand, a
program that is run often but only for short periods of time, such as the
Mercurial command-line client, would be better off with a short startup time
because it likely only runs each chunk of code once, and the fact that the JIT'd
code is faster doesn't overcome the fact that it takes longer to start running.

.NET can't execute IL code directly; it always gets JIT compiled into machine
code, and this takes time. In particular, program startup times are one of the
weak spots of the .NET Framework because much of the code needs to be JIT
compiled. While there are ways to avoid the JIT penalty in static .NET
programs\footnote{Native Image Generation, or
NGEN---\url{http://msdn.microsoft.com/en-us/library/6t9t5wcf.aspx}.}, they
don't work for dynamic programs. Rather than always compile directly to IL,
IronRuby and IronPython will use their own interpreter (found in
\code{Microsoft.Scripting.Interpreter}) that isn't as fast as JIT-compiled code
but takes much less time to get started. The interpreter is also useful in
situations where dynamic code generation is not allowed, such as on mobile
platforms; otherwise the DLR languages would not be able to run at all.

Before execution, the entire expression tree must be wrapped in a function so
that it can be executed. In the DLR, functions are represented as
\code{LambdaExpression} nodes. While in most languages a lambda is an anonymous
function, the DLR has no concept of names; all functions are anonymous. The
\code{LambdaExpression} is unique in that it is the only node type that can be
converted to a \emph{delegate}, which is what .NET calls first-class functions,
using its \code{Compile()} method. A delegate is similar to a C function
pointer---it is simply a handle to a piece of code that can be called.

Initially, the expression tree is wrapped in a \code{LightLambdaExpression},
which can also produce a delegate that can be executed, but rather than
generate IL code (which would then invoke the JIT), it instead compiles the
expression tree to a list of instructions that are then executed on the
interpreter's simple VM. The interpreter is a simple stack-based one;
instructions pop values off of the stack, perform an operation, and then push
the result back on the stack. Each instruction is an instance of a class
derived from \code{Microsoft.Scripting.Interpreter.Instruction} (such as
\code{AddInstruction} or \code{BranchTrueInstruction}) that has properties
describing how many items it takes off of the stack, how many it will put on,
and a \code{Run()} method that executes the instruction by popping and pushing
values on the stack and returning the offset of the next instruction. The
interpreter takes the list of instructions and executes them one by one,
jumping forward or backwards depending on the return value of the \code{Run()}
method.

Once a a piece of code has been executed a certain number of times, it will be
converted to a full \code{LambdaExpression} by calling
\code{LightLambdaExpression.Reduce()}, then compiled to a \code{DynamicMethod}
delegate (on a background thread for a bit of parallelism), and the old
delegate call site will be replaced with the newer, faster one. This greatly
reduces the cost of executing functions that may only be called a few times,
such as the main function of a program, while making commonly called functions
run as fast as possible. By default, the compilation threshold is set at 32
executions, but this can be changed with a command-line option or by the host
program, and can include disabling either compilation or the interpreter 
entirely.

Whether running through the interpreter or compiled to IL, the language's
operations are not hard-coded by the expression tree compiler. Instead, the
compiler generates call sites for each operation that may be dynamic (which is
nearly all of them). These call sites give the objects a chance to implement
dynamic behaviour while still keeping performance high.

\end{aosasect1}

\begin{aosasect1}{Dynamic Call Sites}

In a static .NET language, all of the decisions about what code should be
called are made at compile time. For example, consider the following line of
C\#:

\begin{verbatim}
var z = x + y;
\end{verbatim}

\noindent The compiler knows what the types of `x' and `y' are and whether or not they
can be added. The compiler can emit the proper code for handling overloaded
operators, type conversions, or whatever else might be needed to make the code
run properly, based solely on the static information it knows about the types
involved. Now, consider the following line of Python code:

\begin{verbatim}
z = x + y
\end{verbatim}

The IronPython compiler has \emph{no idea} what this might do when it
encounters it, because it doesn't know what the types of \code{x} and \code{y}
are\footnote{In principle it could, but neither IronPython nor IronRuby do type
inference.}, and even if it did know, the ability of \code{x} and \code{y} to be
added could change at runtime anyway. Instead of emitting the IL code for
adding numbers, the IronPython emits a \emph{call site} that will be resolved
at runtime.

A call site is a placeholder for an operation to be determined at runtime; they
are implemented as instances of the
\code{System.Runtime.CompilerServices.CallSite} class. In a dynamic language
like Ruby or Python, just about every operation has a dynamic component; these
dynamic operations are represented in the expression trees as
\code{DynamicExpression} nodes, which the expression tree compiler knows to
convert to a call site. When a call site is created, it is does not yet know
how to perform the desired operation; however, it is created with an instance
of the proper \emph{call site binder} that is specific to the language in use,
and contains all of the necessary information about how to perform the
operation.

\aosafigure[250pt]{../images/ironlang/callsiteclasses.pdf}{CallSite class diagram}{fig.ironlang.callsite}

Each language will have a different call site binder for each operation, and
the binders often know many different ways to perform an operation depending on
the arguments given to the call site. However, generating these rules is
expensive (in particular, compiling them to a delegate for execution, which
involves invoking the .NET JIT), so the call site has a multi-level \emph{call
site cache} that stores the rules that have already been created for later use.

\aosafigure[250pt]{../images/ironlang/callsites.pdf}{CallSite flowchart}{fig.ironlang.flowchart}

The first level, L0, is the \code{CallSite.Target} property on the call site
instance itself. This stores the most-recently-used rule for this call site;
for a vast number of call sites, this is all that will ever be needed as they
are only ever called with one set of argument types. The call site also has
another cache, L1, that stores a further 10 rules. If \code{Target} is not
valid for this call (for example, if the arguments types are different), the
call site first checks its rules cache to see if it has already created the
proper delegate from a previous call, and reuses that rule instead of creating
a new one.

Storing rules in the cache is driven by the time it takes to actually compile a
new rule compared to the time it takes to check the existing rules. Roughly
speaking, it takes about 10 ns for .NET to execute a type check on a variable
(checking a binary function takes 20 ns, etc.), which is the most common type
of rule predicate. Compiling a simple method to add doubles, on the other hand,
takes about 80 $\mu$s, or three orders of magnitude longer. The size of the caches
is limited to prevent wasting memory storing every rule that gets used at a
call site; for a simple addition, each variation requires about 1 KB of memory.
However, profiling showed that very few call sites ever had more than 10
variations.

Finally, there is the L2 cache, which is stored on the binder instance itself.
The binder instance that is associated with a call site may store some extra
information with it that makes it specific to a call site, but a large number
of call sites aren't unique in any way and can share the same binder instance.
For example, in Python, the basic rules for addition are the same throughout
the program; it depends on the two types on the either side of the \code{+},
and that's it. All of the addition operations in the program can share the same
binder, and if both the L0 and L1 caches miss, the L2 cache contains a much
larger number of recent rules (128) collected from across the entire program.
Even if a call site is on its first execution, there's a good chance it might
already find an appropriate rule in the L2 cache. To ensure that this works
most effectively, IronPython and IronRuby both have a set of canonical binder
instances that are used for common operations like addition.

If the L2 cache misses, the binder is asked to create an \emph{implementation}
for the call site, taking into account the types (and possibly even the values)
of the arguments. In the above example, if \code{x} and \code{y} are doubles
(or another native type), then the implementation simply casts them to doubles
and calls the IL \code{add} instruction. The binder also produces a test that
checks the arguments and ensures they are valid for the implementation.
Together, the implementation and the test make a rule. In most cases, both the
implementation and the test are created and stored as expression
trees\footnote{The call site infrastructure does not depend on expression
trees, however; it can be used with delegates alone.}.

\newpage %% to keep the next para with the code that follows. --ARB Feb 22

\noindent If the expression trees were expressed in C\#, the code would be similar to:

\begin{verbatim}
if(x is double && y is double) {       // check for doubles
      return (double)x + (double)y;    // execute if doubles
 }
 return site.Update(site, x, y);       // not doubles, so find/create another rule 
                                       // for these types
\end{verbatim}

The binder then produces a \emph{delegate} from the expression trees, which
means the rule is compiled to IL and then to machine code. In the case of
adding two numbers, this will likely become a quick type check and then a
machine instruction to add the numbers. Even with all of the machinery
involved, the ultimate end result is only marginally slower than static code.
IronPython and IronRuby also include a set of precompiled rules for common
operations like addition of primitive types, which saves time because they don't
have to be created at runtime, but does cost some extra space on disk.

\end{aosasect1}

\begin{aosasect1}{Meta-Object Protocol}

Besides the language infrastructure, the other key part of the DLR is the
ability for a language (the \emph{host language}) to make dynamic calls on
objects defined in another language (the \emph{source language}). To make this
possible, the DLR must be able to understand what operations are valid on an
object, no matter the language it was written in. Python and Ruby have fairly
similar object models, but JavaScript has a radically different prototype-based
(as opposed class-based) type system. Instead of trying to unify the various
type systems, the DLR treats them all as if they were based on Smalltalk-style
\emph{message passing}.

In a message-passing object-oriented system, objects send messages to other
objects (with parameters, usually), and the object can return another object as
a result. Thus, while each language has its own idea of what an object is, they
can almost all be made equivalent by viewing method calls as messages that are
sent between objects. Of course, even static OO languages fit this model to
some extent; what makes dynamic languages different is that the method being
called does not have to be known at compile time, or even exist on the object
at all (e.g., Ruby's \code{method\_missing}), and the target object usually has
a chance to intercept the message and process it differently if necessary (e.g.,
Python's \code{\_\_getattr\_\_}).

The DLR defines the following messages:

\begin{aosaitemize}

\item \code{{Get|Set|Delete}Member}: operations for manipulating an object's
members

\item \code{{Get|Set|Delete}Index}: operations for indexed objects (such as
arrays or dictionaries)

\item \code{Invoke}, \code{InvokeMember}: invoke an object or member of an
object

\item \code{CreateInstance}: create an instance of an object

\item \code{Convert}: convert an object from one type to another

\item \code{UnaryOperation}, \code{BinaryOperation}: perform operator-based
operations, such as negate (\code{!}) or add (\code{+})

\end{aosaitemize}

Taken together, these operations should be sufficient for implementing just
about any language's object model.

Because the CLR is inherently statically typed, dynamic language objects must
still be represented by static classes. The usual technique is to have a static
class such as \code{PythonObject} and have the actual Python objects be
instances of this class or its subclasses. For reasons of interoperability and
performance, the DLR's mechanism is a lot more complicated. Instead of dealing
with language-specific objects the DLR deals with \emph{meta-objects}, which
are subclasses of \code{System.Dynamic.DynamicMetaObject} and have methods for
handling all of the above messages. Each language has its own subclasses of
\code{DynamicMetaObject} that implement the language's object model, such as
IronPython's \code{MetaPythonObject}. The meta classes also have corresponding
concrete classes that implement the
\code{System.Dynamic.IDynamicMetaObjectProtocol} interface, which is how the
DLR identifies dynamic objects. 

\aosafigure[325pt]{../images/ironlang/idmop.pdf}{IDMOP class diagram}{fig.ironlang.class}
%% QUERY: Please take a close look at diagrams since I resaved them from
%% OmniGraffle. --ARB

From a class that implements \code{IDynamicMetaObjectProtocol}, the DLR can get
a \code{Dynamic\-Meta\-Object} by calling \code{GetMetaObject()}. This
\code{DynamicMetaObject} is provided by the language and implements the binding
functions as required by that object. Each \code{DynamicMetaObject} also has
the value and type, if available, of the underlying object. Finally, a
\code{DynamicMetaObject} stores an expression tree representing the call site
so far and any restrictions on that expression, similar to the call site
binders. 

%% FIXME: DANGER - manual linebreaks ahead - make sure this is still working
%% in the final draft. --ARB
When the DLR is compiling a call to a method on a user-defined class, it first
creates a call site (i.e., an instance of the \code{CallSite} class). The call
site initiates the binding process as described above in ``Dynamic Call
Sites'', which results in it eventually calling \code{GetMetaObject()} on an
instance of \code{OldInstance}\footnote{Python has old-style and new-style
classes, but that's not relevant here.}, which returns a \code{MetaOldInstance}.
Next, a binder is called (\code{PythonGetMemberBinder.Bind()}) which in turn
calls \code{MetaOldInstance.BindGetMember()}; \linebreak
it returns a new \code{DynamicMetaObject} that describes how to look up 
the method name on \linebreak
the object. Then another binder, \code{PythonInvokeBinder.Bind()}, is 
called, which calls \linebreak
\code{MetaOldInstance.BindInvoke()}, wrapping the first
\code{DynamicMetaObject} with a new one representing how to call the method
that was looked up. This includes the original object, the expression tree for
looking up the method name, and \code{DynamicMetaObject}s representing the arguments to
the method.

Once the final \code{DynamicMetaObject} in an expression has been built, its
expression tree and restrictions are used to build a delegate which is then
returned to the call site that initiated the binding. From there the code can
be stored in the call site caches, making operations on objects as fast as
other dynamic calls, and almost as fast as static calls.

Host languages that want to perform dynamic operations on dynamic languages
must derive their binders from \code{DynamicMetaObjectBinder}. The
\code{DynamicMetaObjectBinder} will first ask the target object to bind the
operation (by calling \code{GetMetaObject()} and going through the binding
process described above) before falling back on the host language's binding
semantics. As a result, if an IronRuby object is accessed from an IronPython
program, the binding is first attempted with Ruby (target language) semantics;
if that fails, the \code{DynamicMetaObjectBinder} will fall back on the Python
(host language) semantics. If the object being bound is not dynamic (i.e., it
does not implement \code{IDynamicMetaObjectProvider}), such as classes from the
.NET base class library, then it is accessed with the host language's semantics
using .NET reflection.

Languages do have some freedom in how they implement this; IronPython's
\code{PythonInvokeBinder} does not derive from \code{InvokeBinder} because it
needs to do some extra processing specific to Python objects. As long as it
only deals with Python objects, there are no issues; if it encounters an object
that implements \code{IDynamicMetaObjectProvider} but is not a Python object,
it forwards to a \code{CompatibilityInvokeBinder} class that does inherit from
\code{InvokeBinder} and can handle foreign objects correctly.

If the fallback cannot bind the operation, it doesn't throw an
exception; instead, it returns a \code{DynamicMetaObject} representing the
error. The host language's binder will then handle this in an appropriate
manner for the host language; for example, accessing a missing member on an
IronPython object from a hypothetical JavaScript implementation could return
\code{undefined}, while doing the same to a JavaScript object from IronPython
would raise an \code{AttributeError}.

The ability for languages to work with dynamic objects is rather useless
without the ability to first load and execute code written in other languages,
and for this the DLR provides a common mechanism for hosting other languages.

\end{aosasect1}

\begin{aosasect1}{Hosting}

In addition to providing common language implementation details, the DLR also
provides a shared \emph{hosting interface}. The hosting interface is used by
the host language (usually a static language like C\#) to execute code written
in another language such as Python or Ruby. This is a common technique that
allows end users to extend an application, and the DLR takes it step further by
making it trivial to use any scripting language that has a DLR implementation.
There are four key parts to the hosting interface: the \emph{runtime}, 
\emph{engines}, \emph{sources},  and \emph{scopes}.

The \code{ScriptRuntime} is generally shared amongst all dynamic languages in
an application. The runtime handles all of the current assembly references that
are presented to the loaded languages, provides methods for quick execution of
a file, and provides the methods for creating new engines. For simple scripting
tasks, the runtime is the only interface that needs to be used, but the DLR
also provides classes with more control over how scripts are run.

Usually, only one \code{ScriptEngine} is used for each scripting language. The
DLR's meta-object protocol means that a program can load scripts from multiple
languages, and the objects created by each language can all seamlessly
interoperate. The engine wraps a language-specific \code{LanguageContext} (such
as \code{PythonContext} or \code{RubyContext}) and is used for executing code
from files or strings and performing operations on dynamic objects from
languages that don't natively support the DLR (such as C\# prior to .NET 4).
Engines are thread-safe, and can execute multiple scripts in parallel, as long as
each thread has its own scope. It also provides methods for creating script
sources, which allow for more fine-grained control of script execution. 

A \code{ScriptSource} holds a chunk of code to be executed; it binds a
\code{SourceUnit} object, which holds the actual code, to the
\code{ScriptEngine} that created the source. This class allows code to be
compiled (which produces a \code{CompiledCode} object that can be cached) or
executed directly. If a chunk of code is going to be executed repeatedly, it's
best to compile first, and then execute the compiled code; for scripts that
will only be executed once, it's best to just execute it directly.

Finally, however the code gets to be executed, a \code{ScriptScope} must be
provided for the code to execute in. The scope is used to hold all of script's
variables, and can be pre-loaded with variables from the host, if necessary.
This allows a host to provide custom objects to the script when it starts
running---for example, an image editor may provide a method to access the
pixels of the image the script is working on. Once a script has executed, any
variables it created can be read from the scope. The other main use of scopes is to
provide isolation, so that multiple scripts can be loaded and executed at the
same time without interfering with each other.

It's important to note that all of these classes are provided by the DLR, not
the language; only the \code{LanguageContext} used by the engine comes from the
language implementation. The language context provides all of the
functionality---loading code, creating scopes, compilation, execution, and
operations on dynamic objects---that is needed by a host, and the DLR hosting
classes provide a more usable interface to access that functionality. Because
of this, the same hosting code can be used to host any DLR-based language.

For dynamic language implementations written in C (such as the original Python
and Ruby), special wrapper code must be written to access code not written in
the dynamic language, and it must be repeated for each supported scripting
language. While software like SWIG exists to make this easier, it's still not
trivial to add a Python or Ruby scripting interface to a program and expose its
object model for manipulation by external scripts. For .NET programs, however,
adding scripting is as simple as setting up a runtime, loading the program's
assemblies into the runtime, and using \code{ScriptScope.SetVariable()} to make
the program's objects available to the scripts. Adding support for scripting to
a .NET application can be done in a matter of minutes, which is a huge bonus of
the DLR.

\end{aosasect1}

\begin{aosasect1}{Assembly Layout}

Because of how the DLR evolved from a separate library into part of the CLR,
there are parts that are in the CLR (call sites, expression trees, binders,
code generation, and dynamic meta objects) and parts that are part of
IronLanguages open-source project (hosting, the interpreter, and a few other
bits not discussed here). The parts that are in the CLR are also included in
the IronLanguages project in \code{Microsoft.Scripting.Core}. The DLR parts are
split into two assemblies, \code{Microsoft.Scripting} and
\code{Microsoft.Dynamic}---the former contains the hosting APIs and the latter
contains code for COM interop, the interpreter, and some other pieces common to
dynamic languages.

The languages themselves are split in two as well: \code{IronPython.dll} and
\code{IronRuby.dll} implement the languages themselves (parsers, binders, etc.)
while \code{IronPython.Modules.dll} and \code{IronRuby.Libraries.dll} implement
the portions of the standard library that are implemented in C in the classic
Python and Ruby implementations.

\end{aosasect1}

\begin{aosasect1}{Lessons Learned}

The DLR is a useful example of a language-neutral platform for dynamic
languages built on top of a static runtime. The techniques it uses to achieve
high-performance dynamic code are tricky to implement properly, so the DLR
takes these techniques and makes them available to every dynamic language
implementation.

IronPython and IronRuby are good examples of how to build a language on top of
the DLR. The implementations are very similar because they were developed at
the same time by close teams, yet they still have significant differences in
implementation. Having multiple different languages
co-developed\footnote{IronPython, IronRuby, a prototype JavaScript, and the
mysterious VBx---a fully dynamic version of VB.}, along with C\#'s and VB's
dynamic features, made sure that the DLR design got plenty of testing during
development.

The actual development of IronPython, IronRuby, and the DLR was handled very
differently than most projects within Microsoft at the time---it was a very
agile, iterative development model with continuous integration running from day
one. This enabled them to change very quickly when they had to, which was good
because the DLR became tied into C\#'s dynamic features early in its
development. While the DLR tests are very quick, only taking a dozen seconds or
so, the language tests take far too long to run (the IronPython test suite
takes about 45 minutes, even with parallel execution); improving this would
have improved the iteration speed. Ultimately, these iterations converged on
the current DLR design, which seems overly complicated in parts but fits
together quite nicely in total.

Having the DLR tied to C\# was critically important because it made sure the
DLR had a place and a ``purpose'', but once the C\# dynamic features were done
the political climate changed (coinciding with an economic downturn) and the
Iron languages lost their support within the company. The hosting APIs, for
example, never made it into the .NET Framework (and it's highly unlikely they
ever will); this means that PowerShell 3, which is also based on the DLR, uses
a completely different set of hosting APIs than IronPython and IronRuby,
although all of their objects can still interact as described
above\footnote{Some of the DLR team members went on to work on the C\#
compiler-as-a-service library code-named ``Roslyn'', which bears a striking
resemblance to the IronPython and IronRuby hosting APIs.}. But, thanks to the
wonder of open source licensing, they will continue to survive and even thrive.

\end{aosasect1}

\end{aosachapter}
