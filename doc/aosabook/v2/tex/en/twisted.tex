\begin{aosachapter}{Twisted}{s:twisted}{Jessica McKellar}

Twisted is an event-driven networking engine in Python. It was born in the
early 2000s, when the writers of networked games had few scalable and no
cross-platform libraries, in any language, at their disposal. The authors of
Twisted tried to develop games in the existing networking landscape, struggled,
saw a clear need for a scalable, event-driven, cross-platform networking
framework and decided to make one happen, learning from the mistakes and
hardships of past game and networked application writers.

Twisted supports many common transport and application layer protocols,
including TCP, UDP, SSL/TLS, HTTP, IMAP, SSH, IRC, and FTP.  Like the language
in which it is written, it is ``batteries-included''; Twisted comes with client
and server implementations for all of its protocols, as well as utilities that
make it easy to configure and deploy production-grade Twisted applications from
the command line.

\begin{aosasect1}{Why Twisted?}

In 2000, glyph, the creator of Twisted, was working on a text-based
multiplayer game called Twisted Reality. It was a big mess of threads, 3 per
connection, in Java. There was a thread for input that would block on reads, a
thread for output that would block on some kind of write, and a ``logic'' thread
that would sleep while waiting for timers to expire or events to queue. As
players moved through the virtual landscape and interacted, threads were
deadlocking, caches were getting corrupted, and the locking logic was never
quite right---the use of threads had made the software complicated, buggy, and
hard to scale.

Seeking alternatives, he discovered Python, and in particular Python's
\code{select} module for multiplexing I/O from stream objects like
sockets and pipes\footnote{The Single UNIX Specification, Version 3
  (SUSv3) describes the \code{select} API.}; at the time, Java didn't
expose the operating system's \code{select} interface or any other
asynchronous I/O API\footnote{The \code{java.nio} package for
  non-blocking I/O was added in J2SE 1.4, released in 2002.}. A quick
prototype of the game in Python using \code{select} immediately proved
less complex and more reliable than the threaded version.

An instant convert to Python, \code{select}, and event-driven
programming, glyph wrote a client and server for the game in Python
using the \code{select} API. But then he wanted to do more. Fundamentally,
he wanted to be able to turn network activity into method calls on objects in
the game. What if you could receive email in the game, like the Nethack mailer
daemon? What if every player in the game had a home page? Glyph found himself
needing good Python IMAP and HTTP clients and servers that used
\code{select}.

He first turned to Medusa, a platform developed in the mid-'90s for
writing networking servers in Python based on the \code{asyncore}
module\footnote{\url{http://www.nightmare.com/medusa/}}. \code{asyncore}
is an asynchronous socket handler that builds a dispatcher and
callback interface on top of the operating system's \code{select} API.

This was an inspiring find for glyph, but Medusa had two drawbacks:

\begin{aosaenumerate}

\item It was on its way towards being unmaintained by 2001 when glyph
  started working on Twisted Reality.

\item \code{asyncore} is such a thin wrapper around sockets that
  application programmers are still required to manipulate sockets
  directly. This means portability is still the responsibility of the
  programmer. Additionally, at the time, \code{asyncore}'s Windows
  support was buggy, and glyph knew that he wanted to run a GUI client
  on Windows.

\end{aosaenumerate}

Glyph was facing the prospect of implementing a networking platform himself
and realized that Twisted Reality had opened the door to a problem that was just
as interesting as his game.

Over time, Twisted Reality the game became Twisted the networking platform,
which would do what existing networking platforms in Python didn't:

\begin{aosaitemize}

\item Use event-driven programming instead of multi-threaded
  programming.

\item Be cross-platform: provide a uniform interface to the event
  notification systems exposed by major operating systems.

\item Be ``batteries-included'': provide implementations of popular
  application-layer protocols out of the box, so that Twisted is
  immediately useful to developers.

\item Conform to RFCs, and prove conformance with a robust test suite.

\item Make it easy to use multiple networking protocols together.

\item Be extensible.

\end{aosaitemize}

\end{aosasect1}

\begin{aosasect1}{The Architecture of Twisted}

Twisted is an event-driven networking engine. Event-driven programming is so
integral to Twisted's design philosophy that it is worth taking a moment to
review what exactly event-driven programming means.

Event-driven programming is a programming paradigm in which program flow is
determined by external events. It is characterized by an event loop and the use
of callbacks to trigger actions when events happen. Two other common
programming paradigms are (single-threaded) synchronous and multi-threaded
programming.

\aosafigure{../images/twisted/threading_models.png}{Threading models}{fig:twisted:threadingmodels}

Let's compare and contrast single-threaded, multi-threaded, and
event-driven programming models with an example.
\aosafigref{fig:twisted:threadingmodels} shows the work done by a
program over time under these three models. The program has three tasks to
complete, each of which blocks while waiting for I/O to finish. Time
spent blocking on I/O is greyed out.

In the single-threaded synchronous version of the program, tasks are
performed serially. If one task blocks for a while on I/O, all of the other
tasks have to wait until it finishes and they are executed in turn. This
definite order and serial processing are easy to reason about, but the program
is unnecessarily slow if the tasks don't depend on each other, yet still have to
wait for each other.

In the threaded version of the program, the three tasks that block while doing
work are performed in separate threads of control. These threads are managed by
the operating system and may run concurrently on multiple processors or
interleaved on a single processor. This allows progress to be made by some
threads while others are blocking on resources. This is often more
time-efficient than the analogous synchronous program, but one has to write code
to protect shared resources that could be accessed concurrently from multiple
threads. Multi-threaded programs can be harder to reason about because one now
has to worry about thread safety via process serialization (locking),
reentrancy, thread-local storage, or other mechanisms, which when implemented
improperly can lead to subtle and painful bugs.

The event-driven version of the program interleaves the execution of the three
tasks, but in a single thread of control. When performing I/O or other expensive
operations, a callback is registered with an event loop, and then execution
continues while the I/O completes. The callback describes how to handle an event
once it has completed. The event loop polls for events and dispatches them as
they arrive, to the callbacks that are waiting for them. This allows the program
to make progress when it can without the use of additional threads. Event-driven
programs can be easier to reason about than multi-threaded programs because the
programmer doesn't have to worry about thread safety.

The event-driven model is often a good choice when there are:

\begin{aosaenumerate}

\item many tasks, that are\dots
\item largely independent (so they don't have to communicate with or
  wait on each other), and\ldots
\item some of these tasks block while waiting on events.

\end{aosaenumerate}

It is also a good choice when an application has to share mutable data
between tasks, because no synchronization has to be performed.

Networking applications often have exactly these properties, which is what
makes them such a good fit for the event-driven programming model.

\begin{aosasect2}{Reusing Existing Applications}

Many popular clients and servers for various networking protocols already
existed when Twisted was created. Why did glyph not just use Apache, IRCd, BIND,
OpenSSH, or any of the other pre-existing applications whose clients and servers
would have to get re-implemented from scratch for Twisted?

The problem is that all of these server implementations have networking code
written from scratch, typically in C, with application code coupled directly to
the networking layer. This makes them very difficult to use as libraries. 
They have to be treated as black boxes when used together, giving a developer no
chance to reuse code if he or she wanted to expose the same data over multiple
protocols. Additionally, the server and client implementations are often
separate applications that don't share code. Extending these applications and
maintaining cross-platform client-server compatibility is harder than it needs
to be.

With Twisted, the clients and servers are written in Python using a
consistent interface. This makes it is easy to write new clients and servers, to
share code between clients and servers, to share application logic between
protocols, and to test one's code.

\end{aosasect2}

\begin{aosasect2}{The Reactor}

Twisted implements the \emph{reactor} design pattern, which describes demultiplexing
and dispatching events from multiple sources to their handlers in a
single-threaded environment.

The core of Twisted is the reactor event loop. The reactor knows about network,
file system, and timer events. It waits on and then handles these events,
abstracting away platform-specific behavior and presenting interfaces to make
responding to events anywhere in the network stack easy.

The reactor essentially accomplishes:

\begin{verbatim}
while True:
    timeout = time_until_next_timed_event()
    events = wait_for_events(timeout)
    events += timed_events_until(now())
    for event in events:
        event.process()
\end{verbatim}

A reactor based on the \code{poll} API\footnote{The Single UNIX
  Specification, Version 3 (SUSv3) describes the \code{poll} API.} is
the current default on all platforms. Twisted additionally supports a
number of platform-specific high-volume multiplexing
APIs. Platform-specific reactors include the KQueue reactor based on
FreeBSD's \code{kqueue} mechanism, an \code{epoll}-based reactor for
systems supporting the \code{epoll} interface (currently Linux 2.6),
and an IOCP reactor based on Windows Input/Output Completion Ports.

Examples of polling implementation-dependent details that Twisted takes care
of include:

\begin{aosaitemize}

\item Network and filesystem limits.

\item Buffering behavior.

\item How to detect a dropped connection.

\item The values returned in error cases.

\end{aosaitemize}

Twisted's reactor implementation also takes care of using the underlying
non-blocking APIs correctly and handling obscure edge cases correctly. Python
doesn't expose the IOCP API at all, so Twisted maintains its own
implementation.

\end{aosasect2}

\begin{aosasect2}{Managing Callback Chains}

Callbacks are a fundamental part of event-driven programming and are the way
that the reactor indicates to an application that events have completed. As
event-driven programs grow, handling both the success and error cases for the
events in one's application becomes increasingly complex. Failing to register an
appropriate callback can leave a program blocking on event processing that will
never happen, and errors might have to propagate up a chain of callbacks from
the networking stack through the layers of an application.

Let's examine some of the pitfalls of event-driven programs by comparing
synchronous and asynchronous versions of a toy URL fetching utility in
Python-like pseudo-code:

\begin{figure}[h!]
\centering
\begin{minipage}[t]{0.4\linewidth}

Synchronous URL fetcher:

\centering
\begin{verbatim}
import getPage


def processPage(page):
    print page

def logError(error):
    print error

def finishProcessing(value):
    print "Shutting down..."
    exit(0)

url = "http://google.com"
try:
    page = getPage(url)
    processPage(page)
except Error, e:
    logError(error)
finally:
    finishProcessing()
\end{verbatim}

\end{minipage}
\hspace{0.01cm}
\begin{minipage}[t]{0.4\linewidth}

Asynchronous URL fetcher:

\centering

\begin{verbatim}
from twisted.internet import reactor
import getPage

def processPage(page):
    print page
    finishProcessing()

def logError(error):
    print error
    finishProcessing()

def finishProcessing(value):
    print "Shutting down..."
    reactor.stop()

url = "http://google.com"
# getPage takes: url, 
#    success callback, error callback
getPage(url, processPage, logError)

reactor.run()
\end{verbatim}
\end{minipage}
\end{figure}

\noindent In the asynchronous URL fetcher,
\code{reactor.run()} starts the reactor event loop. 
In both the synchronous and asynchronous versions, a hypothetical
\code{getPage} function does the work of page
retrieval. \code{processPage} is invoked if the retrieval is successful,
and \code{logError} is invoked if an \code{Exception} is raised
while attempting to retrieve the page. In either case,
\code{finishProcessing} is called afterwards.

The callback to \code{logError} in the asynchronous version mirrors the
\code{except} part of the \code{try/except} block in the synchronous
version. The callback to \code{processPage} mirrors \code{else}, and
the unconditional callback to \code{finishProcessing} mirrors
\code{finally}.

In the synchronous version, by virtue of the structure of a
\code{try/except} block exactly one of \code{logError} and
\code{processPage} is called, and \code{finishProcessing} is always
called once; in the asynchronous version it is the programmer's responsibility
to invoke the correct chain of success and error callbacks. If, through
programming error, the call to \code{finishProcessing} were left out of
\code{processPage} or \code{logError} along their respective
callback chains, the reactor would never get stopped and the program would run
forever.

This toy example hints at the complexity frustrating programmers during the
first few years of Twisted's development. Twisted responded to this complexity
by growing an object called a \code{Deferred}.

\begin{aosasect3}{Deferreds}

The \code{Deferred} object is an abstraction of the idea of a result that
doesn't exist yet. It also helps manage the callback chains for this
result. When returned by a function, a \code{Deferred} is a promise
that the function will have a result \emph{at some point}. That single
returned \code{Deferred} contains references to all of the callbacks
registered for an event, so only this one object needs to be passed
between functions, which is much simpler to keep track of than
managing callbacks individually.

\code{Deferred}s have a pair of callback chains, one for success
(callbacks) and one for errors (errbacks). \code{Deferred}s start out with
two empty chains. One adds pairs of callbacks and errbacks to handle successes
and failures at each point in the event processing. When an asynchronous result
arrives, the \code{Deferred} is ``fired'' and the appropriate callbacks or
errbacks are invoked in the order in which they were added.

Here is a version of the asynchronous URL fetcher pseudo-code which
uses \code{Deferred}s:

\begin{verbatim}
from twisted.internet import reactor
import getPage

def processPage(page):
    print page

def logError(error):
    print error

def finishProcessing(value):
    print "Shutting down..."
    reactor.stop()

url = "http://google.com"
deferred = getPage(url) # getPage returns a Deferred
deferred.addCallbacks(processPage, logError)
deferred.addBoth(finishProcessing)

reactor.run()
\end{verbatim}

In this version, the same event handlers are invoked, but they are all
registered with a single \code{Deferred} object instead of spread out in
the code and passed as arguments to \code{getPage}.

The \code{Deferred} is created with two stages of callbacks. First,
\code{addCallbacks} adds the \code{processPage} callback and
\code{logError} errback to the first stage of their respective
chains. Then \code{addBoth} adds \code{finishProcessing} to the second
stage of both chains. Diagrammatically, the callback chains look like
\aosafigref{fig:twisted:callback}.

\aosafigure{../images/twisted/deferred.png}{Callback chains}{fig:twisted:callback}

\code{Deferred}s can only be fired once; attempting to re-fire them
will raise an \code{Exception}. This gives \code{Deferred}s
semantics closer to those of the \code{try/except} blocks of their synchronous
cousins, which makes processing the asynchronous events easier to reason about
and avoids subtle bugs caused by callbacks being invoked more or less than once
for a single event.

Understanding \code{Deferred}s is an important part of understanding
the flow of Twisted programs. However, when using the high-level abstractions
Twisted provides for networking protocols, one often doesn't have to use
\code{Deferred}s directly at all.

The \code{Deferred} abstraction is powerful and has been borrowed by
many other event-driven platforms, including jQuery, Dojo, and Mochikit.

\end{aosasect3}

\end{aosasect2}

\begin{aosasect2}{Transports}

Transports represent the connection between two endpoints communicating over
a network. Transports are responsible for describing connection details, like
being stream- or datagram-oriented, flow control, and reliability. TCP, UDP, and
Unix sockets are examples of transports. They are designed to be ``minimally
functional units that are maximally reusable'' and are decoupled from protocol
implementations, allowing for many protocols to utilize the same type of
transport. Transports implement the \code{ITransport} interface, which has
the following methods:

\vspace{8pt}
\begin{tabular}{lp{9cm}}
\code{write} & Write some data to the physical connection, in sequence, in a non-blocking fashion. \\
\code{writeSequence} & Write a list of strings to the physical connection. \\
\code{loseConnection} & Write all pending data and then close the connection. \\
\code{getPeer} & Get the remote address of this connection. \\
\code{getHost} & Get the address of this side of the connection. \\
\end{tabular}
\vspace{8pt}

Decoupling transports from procotols also makes testing the two layers
easier. A mock transport can simply write data to a string for inspection.

\end{aosasect2}

\begin{aosasect2}{Protocols}

Procotols describe how to process network events asynchronously. HTTP, DNS,
and IMAP are examples of application protocols. Protocols implement the
\code{IProtocol} interface, which has the following methods:

\vspace{8pt}
\begin{tabular}{ll}
\code{makeConnection} & Make a connection to a transport and a server. \\
\code{connectionMade} & Called when a connection is made. \\
\code{dataReceived} & Called whenever data is received. \\
\code{connectionLost} & Called when the connection is shut down. \\
\end{tabular}
\vspace{8pt}

The relationship between the reactor, protocols, and transports is best
illustrated with an example. Here are complete implementations of an echo
server and client, first the server:

\begin{verbatim}
from twisted.internet import protocol, reactor

class Echo(protocol.Protocol):
    def dataReceived(self, data):
        # As soon as any data is received, write it back
        self.transport.write(data)

class EchoFactory(protocol.Factory):
    def buildProtocol(self, addr):
        return Echo()

reactor.listenTCP(8000, EchoFactory())
reactor.run()
\end{verbatim}

\noindent And the client:

\begin{verbatim}
from twisted.internet import reactor, protocol

class EchoClient(protocol.Protocol):
   def connectionMade(self):
       self.transport.write("hello, world!")

   def dataReceived(self, data):
       print "Server said:", data
       self.transport.loseConnection()

   def connectionLost(self, reason):
       print "connection lost"

class EchoFactory(protocol.ClientFactory):
   def buildProtocol(self, addr):
       return EchoClient()

   def clientConnectionFailed(self, connector, reason):
       print "Connection failed - goodbye!"
       reactor.stop()

   def clientConnectionLost(self, connector, reason):
       print "Connection lost - goodbye!"
       reactor.stop()

reactor.connectTCP("localhost", 8000, EchoFactory())
reactor.run()
\end{verbatim}

Running the server script starts a TCP server listening for connections on
port 8000. The server uses the \code{Echo} protocol, and data is written
out over a TCP transport. Running the client makes a TCP connection to the
server, echoes the server response, and then terminates the connection and stops
the reactor. Factories are used to produce instances of protocols for both sides
of the connection. The communication is asynchronous on both sides;
\code{connectTCP} takes care of registering callbacks with the reactor to
get notified when data is available to read from a socket.

\end{aosasect2}

\begin{aosasect2}{Applications}

Twisted is an engine for producing scalable, cross-platform network servers
and clients. Making it easy to deploy these applications in a standardized
fashion in production environments is an important part of a platform like this
getting wide-scale adoption.

To that end, Twisted developed the Twisted application infrastructure, a
re-usable and configurable way to deploy a Twisted application. It allows a
programmer to avoid boilerplate code by hooking an application into existing
tools for customizing the way it is run, including daemonization, logging, using
a custom reactor, profiling code, and more.

The application infrastructure has four main parts: Services, Applications,
configuration management (via TAC files and plugins), and the \code{twistd}
command-line utility. To illustrate this infrastructure, we'll turn the echo
server from the previous section into an Application.

\begin{aosasect3}{Service}

A Service is anything that can be started and stopped and which adheres to
the \code{IService} interface. Twisted comes with service implementations
for TCP, FTP, HTTP, SSH, DNS, and many other protocols. Many Services can
register with a single application.

The core of the \code{IService} interface is:

\vspace{8pt}
\begin{tabular}{lp{9cm}}
\code{startService} & Start the service. This might include
  loading configuration data, setting up database connections, or
  listening on a port.\\
\code{stopService} & Shut down the service. This might include
  saving state to disk, closing database connections, or stopping
  listening on a port.\\
\end{tabular}
\vspace{8pt}

Our echo service uses TCP, so we can use Twisted's default
\code{TCPServer} implementation of this \code{IService}
interface.

\end{aosasect3}

\begin{aosasect3}{Application}

An Application is the top-level service that represents the entire Twisted
application. Services register themselves with an Application, and the \code{twistd}
deployment utility described below searches for and runs Applications.

We'll create an echo Application with which the echo Service can
register.

\end{aosasect3}

\begin{aosasect3}{TAC Files}

When managing Twisted applications in a regular Python file, the developer is
responsible for writing code to start and stop the reactor and to configure the
application. Under the Twisted application infrastructure, protocol
implementations live in a module, Services using those protocols are registered
in a Twisted Application Configuration (TAC) file, and the reactor and
configuration are managed by an external utility.

To turn our echo server into an echo application, we can follow a simple
algorithm:

\begin{aosaenumerate}
  \item Move the Protocol parts of the echo server into their own
    module.
  \item Inside a TAC file:

  \begin{aosaenumerate2}
    \item Create an echo Application.
    \item Create an instance of the \code{TCPServer} Service which will use our
    \code{EchoFactory}, and register it with the Application.
  \end{aosaenumerate2}
\end{aosaenumerate}

The code for managing the reactor will be taken care of by
\code{twistd}, discussed below. The application code ends up looking like
this:

The \code{echo.py} file:

\begin{verbatim}
from twisted.internet import protocol, reactor

class Echo(protocol.Protocol):
    def dataReceived(self, data):
        self.transport.write(data)

class EchoFactory(protocol.Factory):
    def buildProtocol(self, addr):
    return Echo()
\end{verbatim}

\noindent The \code{echo\_server.tac} file:

\begin{verbatim}
from twisted.application import internet, service
from echo import EchoFactory

application = service.Application("echo")
echoService = internet.TCPServer(8000, EchoFactory())
echoService.setServiceParent(application)
\end{verbatim}

\end{aosasect3}

\begin{aosasect3}{twistd}

\code{twistd} (pronounced ``twist-dee'') is a cross-platform utility for
deploying Twisted applications. It runs TAC files and handles starting and
stopping an application. As part of Twisted's batteries-included approach to
network programming, \code{twistd} comes with a number of useful configuration flags,
including daemonizing the application, the location of log files, dropping
privileges, running in a chroot, running under a non-default reactor, or even
running the application under a profiler.

We can run our echo server Application with:

\begin{verbatim}
$ twistd -y echo_server.tac
\end{verbatim}

In this simplest case, \code{twistd} starts a daemonized instance of the
application, logging to \linebreak
\code{twistd.log}. After starting and stopping the
application, the log looks like this:

\begin{verbatim}
2011-11-19 22:23:07-0500 [-] Log opened.
2011-11-19 22:23:07-0500 [-] twistd 11.0.0 (/usr/bin/python 2.7.1) starting up.
2011-11-19 22:23:07-0500 [-] reactor class: twisted.internet.selectreactor.SelectReactor.
2011-11-19 22:23:07-0500 [-] echo.EchoFactory starting on 8000
2011-11-19 22:23:07-0500 [-] Starting factory <echo.EchoFactory instance at 0x12d8670>
2011-11-19 22:23:20-0500 [-] Received SIGTERM, shutting down.
2011-11-19 22:23:20-0500 [-] (TCP Port 8000 Closed)
2011-11-19 22:23:20-0500 [-] Stopping factory <echo.EchoFactory instance at 0x12d8670>
2011-11-19 22:23:20-0500 [-] Main loop terminated.
2011-11-19 22:23:20-0500 [-] Server Shut Down.
\end{verbatim}

Running a service using the Twisted application infrastructure allows
developers to skip writing boilerplate code for common service functionalities
like logging and daemonization. It also establishes a standard command line
interface for deploying applications.

\end{aosasect3}

\begin{aosasect3}{Plugins}

An alternative to the TAC-based system for running Twisted applications is
the plugin system. While the TAC system makes it easy to register simple
hierarchies of pre-defined services within an application configuration file,
the plugin system makes it easy to register custom services as subcommands of
the \code{twistd} utility, and to extend the command-line interface to an
application.

Using this system:

\begin{aosaenumerate}

\item Only the plugin API is required to remain stable, which makes it
  easy for third-party developers to extend the software.

\item Plugin discoverability is codified. Plugins can be loaded and
  saved when a program is first run, re-discovered each time the
  program starts up, or polled for repeatedly at runtime, allowing the
  discovery of new plugins installed after the program has started.

\end{aosaenumerate}

To extend a program using the Twisted plugin system, all one has to do is
create objects which implement the \code{IPlugin} interface and put them
in a particular location where the plugin system knows to look for them.

Having already converted our echo server to a Twisted application,
transformation into a Twisted plugin is straightforward. Alongside the
\code{echo} module from before, which contains the \code{Echo}
protocol and \code{EchoFactory} definitions, we add a directory called
\code{twisted}, containing a subdirectory called \code{plugins},
containing our echo plugin definition. This plugin will allow us to start an
echo server and specify the port to use as arguments to the \code{twistd} utility:

\begin{verbatim}
from zope.interface import implements

from twisted.python import usage
from twisted.plugin import IPlugin
from twisted.application.service import IServiceMaker
from twisted.application import internet

from echo import EchoFactory

class Options(usage.Options):
    optParameters = [["port", "p", 8000, "The port number to listen on."]]

class EchoServiceMaker(object):
    implements(IServiceMaker, IPlugin)
    tapname = "echo"
    description = "A TCP-based echo server."
    options = Options

    def makeService(self, options):
        """
        Construct a TCPServer from a factory defined in myproject.
        """
        return internet.TCPServer(int(options["port"]), EchoFactory())

serviceMaker = EchoServiceMaker()
\end{verbatim}

Our echo server will now show up as a server option in the output of
\code{twistd\,--help}, and running \code{twistd echo\,--port=1235}
will start an echo server on port 1235.

Twisted comes with a pluggable authentication system for servers called
\code{twisted.cred}, and a common use of the plugin system is to add an
authentication pattern to an application. One can use 
\code{twisted.cred AuthOptionMixin} to add command-line support for various kinds of
authentication off the shelf, or to add a new kind. For example, one could add
authentication via a local Unix password database or an LDAP server using the
plugin system.

\code{twistd} comes with plugins for many of Twisted's supported protocols,
which turns the work of spinning up a server into a single command. Here are
some examples of \code{twistd} servers that ship with Twisted:

\begin{aosadescription}
\item{\code{twistd web --port 8080 --path .}} \\ Run an HTTP server on
  port 8080, serving both static and dynamic content out of the
  current working directory.

\item{\code{twistd dns -p 5553 --hosts-file=hosts}} \\ Run a DNS server
  on port 5553, resolving domains out of a file called \code{hosts} in the
  format of \code{/etc/hosts}.

\item{\code{sudo twistd conch -p tcp:2222}}  \\ Run an ssh server on port
  2222. ssh keys must be set up independently.

\item{\code{twistd mail -E -H localhost -d localhost=emails}} \\ Run an
  ESMTP POP3 server, accepting email for localhost and saving it to
  the \code{emails} directory.
\end{aosadescription}

\code{twistd} makes it easy to spin up a server for testing clients,
but it is also pluggable, production-grade code.

In that respect, Twisted's application deployment mechanisms via TAC files,
plugins, and \code{twistd} have been a success. However, anecdotally, most
large Twisted deployments end up having to rewrite some of these management and
monitoring facilities; the architecture does not quite expose what system
administrators need. This is a reflection of the fact that Twisted has not
historically had much architectural input from system administrators---the
people who are experts at deploying and maintaining applications.

Twisted would be well-served to more aggressively solicit feedback from
expert end users when making future architectural decisions in this space.

\end{aosasect3}

\end{aosasect2}

\end{aosasect1}

\begin{aosasect1}{Retrospective and Lessons Learned}

Twisted recently celebrated its 10th anniversary. Since its inception,
inspired by the networked game landscape of the early 2000s, it has largely
achieved its goal of being an extensible, cross-platform, event-driven
networking engine. Twisted is used in production environments at companies from
Google and Lucasfilm to Justin.TV and the Launchpad software collaboration
platform. Server implementations in Twisted are the core of numerous other open
source applications, including BuildBot, BitTorrent, and Tahoe-LAFS.

Twisted has had few major architectural changes since its initial
development. The one crucial addition was \code{Deferred}, as
discussed above, for managing pending results and their callback chains.

There was one important removal, which has almost no footprint in the current
implementation: Twisted Application Persistence.

\begin{aosasect2}{Twisted Application Persistence}

Twisted Application Persistence (TAP) was a way of keeping an application's
configuration and state in a pickle. Running an application using this scheme
was a two-step process:

\begin{aosaenumerate}

\item Create the pickle that represents an Application, using the now
  defunct \code{mktap} utility.

\item Use \code{twistd} to unpickle and run the Application.

\end{aosaenumerate}

This process was inspired by Smalltalk images, an aversion to the
proliferation of seemingly ad hoc configuration languages that were hard to
script, and a desire to express configuration details in Python.

TAP files immediately introduced unwanted complexity. Classes would change in
Twisted without instances of those classes getting changed in the pickle. Trying
to use class methods or attributes from a newer version of Twisted on the
pickled object would crash the application. The notion of ``upgraders'' that would
upgrade pickles to new API versions was introduced, but then a matrix of
upgraders, pickle versions, and unit tests had to be maintained to cover all
possible upgrade paths, and comprehensively accounting for all interface changes
was still hard and error-prone.

TAPs and their associated utilities were abandoned and then eventually
removed from Twisted and replaced with TAC files and plugins. TAP was
backronymed to Twisted Application Plugin, and few traces of the failed pickling
system exist in Twisted today.

The lesson learned from the TAP fiasco was that to have reasonable
maintainability, persistent data needs an explicit schema. More generally, it
was a lesson about adding complexity to a project: when considering introducing
a novel system for solving a problem, make sure the complexity of that solution
is well understood and tested and that the benefits are clearly worth the added
complexity before committing the project to it.

\end{aosasect2}

\begin{aosasect2}{web2: a lesson on rewrites}

While not primarily an architectural decision, a project management decision
about rewriting the Twisted Web implementation has had long-term ramifications
for Twisted's image and the maintainers' ability to make architectural
improvements to other parts of the code base, and it deserves a short
discussion.

In the mid-2000s, the Twisted developers decided to do a full rewrite of the
\code{twisted.web} APIs as a separate project in the Twisted code base
called \code{web2}. \code{web2} would contain numerous improvements
over \code{twisted.web}, including full HTTP 1.1 support and a streaming
data API.

\code{web2} was labelled as experimental, but ended up getting used by
major projects anyway and was even accidentally released and packaged by
Debian. Development on \code{web} and \code{web2} continued
concurrently for years, and new users were perennially frustrated by the
side-by-side existence of both projects and a lack of clear messaging about
which project to use. The switchover to \code{web2} never happened, and in
2011 \code{web2} was finally removed from the code base and the
website. Some of the improvements from \code{web2} are slowly getting
ported back to \code{web}.

Partially because of \code{web2}, Twisted developed a reputation for
being hard to navigate and structurally confusing to newcomers. Years later, the
Twisted community still works hard to combat this image.

The lesson learned from \code{web2} was that rewriting a project from
scratch is often a bad idea, but if it has to happen make sure that the
developer community understands the long-term plan, and that the user community
has one clear choice of implementation to use during the rewrite.

If Twisted could go back and do \code{web2} again, the developers would
have done a series of backwards-compatible changes and deprecations to
\code{twisted.web} instead of a rewrite.

\end{aosasect2}

\begin{aosasect2}{Keeping Up with the Internet}

The way that we use the Internet continues to evolve. The decision to
implement many protocols as part of the core software burdens Twisted with
maintaining code for all of those protocols.  Implementations have to evolve
with changing standards and the adoption of new protocols while maintaining a
strict backwards-compatibility policy.

Twisted is primarily a volunteer-driven project, and the limiting factor for
development is not community enthusiasm, but rather volunteer time. For example,
RFC 2616 defining HTTP 1.1 was released in 1999, work began on adding HTTP 1.1
support to Twisted's HTTP protocol implementations in 2005, and the work was
completed in 2009.  Support for IPv6, defined in RFC 2460 in 1998, is in
progress but unmerged as of 2011.

Implementations also have to evolve as the interfaces exposed by
supported operating systems change. For example, the \code{epoll}
event notification facility was added to Linux 2.5.44 in 2002, and
Twisted grew an \code{epoll}-based reactor to take advantage of this new
API. In 2007, Apple released OS~10.5 Leopard with a \code{poll}
implementation that didn't support devices, which was buggy enough
behavior for Apple to not expose \code{select.poll} in its build of
Python\footnote{\url{http://twistedmatrix.com/trac/ticket/4173}}.
Twisted has had to work around this issue and document it for users
ever since.

Sometimes, Twisted development doesn't keep up with the changing networking
landscape, and enhancements are moved to libraries outside of the core
software. For example, the Wokkel project\footnote{\url{http://wokkel.ik.nu/}}, a collection of enhancements to
Twisted's Jabber/XMPP support, has lived as a to-be-merged independent project
for years without a champion to oversee the merge. An attempt was made to
add WebSockets to Twisted as browsers began to adopt support for the new
protocol in 2009, but development moved to external projects after a decision
not to include the protocol until it moved from an IETF draft to a standard.

All of this being said, the proliferation of libraries and add-ons is a
testament to Twisted's flexibility and extensibility. A strict test-driven
development policy and accompanying documentation and coding standards help the
project avoid regressions and preserve backwards compatibility while maintaining
a large matrix of supported protocols and platforms. It is a mature, stable
project that continues to have very active development and adoption.

Twisted looks forward to being the engine of your Internet for another ten
years.

\end{aosasect2}

\end{aosasect1}

\end{aosachapter}
