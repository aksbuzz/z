\begin{aosachapter}{SocialCalc}{s:socialcalc}{Audrey Tang}

The history of spreadsheets spans more than 30 years.  The first
spreadsheet program, VisiCalc, was conceived by Dan Bricklin in 1978
and shipped in 1979.  The original concept was quite straightforward:
a table that spans infinitely in two dimensions, its cells populated
with text, numbers, and formulas.  Formulas are composed of normal
arithmetic operators and various built-in functions, and each formula
can use the current contents of other cells as values.

Although the metaphor was simple, it had many applications:
accounting, inventory, and list management
are just a few.  The possibilities were practically limitless.  All
these uses made VisiCalc into the first ``killer app'' of the
personal computer era.

In the decades that followed successors like Lotus 1-2-3 and Excel
made incremental improvements, but the core metaphor stayed the same.
Most spreadsheets were stored as on-disk files, and loaded into memory
when opened for editing.  Collaboration was particularly hard under
the file-based model:

\begin{aosaitemize}

  \item Each user needed to install a version of the spreadsheet editor.

  \item E-mail ping-pong, shared folders, or setting up a dedicated
  version-control system all added bookkeeping overhead.

  \item Change tracking was limited; for example, Excel does not preserve
  history for formatting changes and cell comments.

  \item Updating formatting or formulas in templates required
  painstaking changes to existing spreadsheet files that used that
  template.

\end{aosaitemize}

Fortunately, a new collaboration model emerged to address these issues
with elegant simplicity.  It is the wiki model, invented by Ward
Cunningham in 1994, and popularized by Wikipedia in the early 2000s.

Instead of files, the wiki model features server-hosted pages,
editable in the browser without requiring special software.  Those
hypertext pages can easily link to each other, and even include
portions of other pages to form a larger page.  All participants view
and edit the latest version by default, with revision history
automatically managed by the server.

Inspired by the wiki model, Dan Bricklin started working on WikiCalc
in 2005.  It aims to combine the authoring ease and multi-person
editing of wikis with the familiar visual formatting and calculating
metaphor of spreadsheets.

\begin{aosasect1}{WikiCalc}

The first version of WikiCalc (\aosafigref{fig.soc.screenshot}) had
several features that set it apart from other spreadsheets at the
time:

\begin{aosaitemize}

  \item Plain text, HTML, and wiki-style markup rendering for text data.

  \item Wiki-style text that includes commands to insert links, images,
  and values from cell references.

  \item Formula cells may reference values of other WikiCalc pages
  hosted on other websites.

  \item Ability to create output to be embedded in other web pages, both
  static and live data.

  \item Cell formatting with access to CSS style attributes and CSS
  classes.

  \item Logging of all edit operations as an audit trail.

  \item Wiki-like retention of each new version of a page with roll-back
  capability.

\end{aosaitemize}

\aosafigure[300pt]{../images/socialcalc/wikicalc-screenshot.png}{WikiCalc 1.0 Interface}{fig.soc.screenshot}

\aosafigure[250pt]{../images/socialcalc/wikicalc-components.png}{WikiCalc Components}{fig.soc.comp}

\aosafigure[300pt]{../images/socialcalc/wikicalc-flow.png}{WikiCalc Flow}{fig.soc.flow}

WikiCalc 1.0's internal architecture (\aosafigref{fig.soc.comp}) and
information flow (\aosafigref{fig.soc.flow}) were deliberately simple,
but nevertheless powerful.  The ability to compose a master
spreadsheet from several smaller spreadsheets proved particularly
handy.  For example, imagine a scenario where each salesperson keeps
numbers in a spreadsheet page.  Each sales manager then rolls up their
reps' numbers into a regional spreadsheet, and the VP of sales then
rolls up the regional numbers into a top-level spreadsheet.

Each time one of the individual spreadsheets is updated, all the
roll-up spreadsheets can reflect the update.  If someone wants further
detail, they simply click through to view the spreadsheet behind the
spreadsheet.  This roll-up capability eliminates the redundant and
error-prone effort of updating numbers in multiple places, and ensures
all views of the information stay fresh.

To ensure the recalculations are up-to-date, WikiCalc adopted a
thin-client design, keeping all the state information on the server
side.  Each spreadsheet is represented on the browser as a
\code{{\textless}table{\textgreater}} element; editing a cell will
send an \code{ajaxsetcell} call to the server, and the server then
tells the browser which cells need updating.

Unsurprisingly, this design depends on a fast connection between the
browser and the server.  When the latency is high, users will start to
notice the frequent appearance of ``Loading\ldots'' messages between
updating a cell and seeing its new contents as shown in
\aosafigref{fig.soc.load}. This is especially a problem for users
interactively editing formulas by tweaking the input and expecting to
see results in real time.

\aosafigure[250pt]{../images/socialcalc/wikicalc-loading.png}{Loading Message}{fig.soc.load}

Moreover, because the \code{{\textless}table{\textgreater}} element
had the same dimensions as the spreadsheet, a $100{\times}100$ grid
would create 10,000 \code{{\textless}td{\textgreater}} DOM objects,
which strains the memory resource of browsers, further limiting the
size of pages.

Due to these shortcomings, while WikiCalc was useful as a stand-alone
server running on localhost, it was not very practical to embed as part
of web-based content management systems.

In 2006, Dan Bricklin teamed up with Socialtext to start developing
SocialCalc, a ground-up rewrite of WikiCalc in Javascript based on
some of the original Perl code.

This rewrite was aimed at large, distributed collaborations, and sought
to deliver a look and feel more like that of a desktop app.  Other design
goals included:

\begin{aosaitemize}

  \item Capable of handling hundreds of thousands of cells.

  \item Fast turnaround time for edit operations.

  \item Client-side audit trail and undo/redo stack.

  \item Better use of Javascript and CSS to provide full-fledged layout
  functionality.

  \item Cross-browser support, despite the more extensive use of
  responsive Javascript.

\end{aosaitemize}

After three years of development and various beta releases, Socialtext
released SocialCalc 1.0 in 2009, successfully meeting the design
goals.  Let's now take a look at the architecture of the SocialCalc
system.

\end{aosasect1}

\begin{aosasect1}{SocialCalc}

\aosafigure[225pt]{../images/socialcalc/socialcalc-screenshot.png}{SocialCalc Interface}{fig.soc.action}

\aosafigref{fig.soc.action} and \aosafigref{fig.soc.class} show
SocialCalc's interface and classes respectively.  Compared to
WikiCalc, the server's role has been greatly reduced.  Its only
responsibility is responding to HTTP GETs by serving entire
spreadsheets serialized in the save format; once the browser receives
the data, all calculations, change tracking and user interaction are
now implemented in Javascript.

\aosafigure[325pt]{../images/socialcalc/socialcalc-class-diagram.png}{SocialCalc Class Diagram}{fig.soc.class}

The Javascript components were designed with a layered MVC
(Model/View/Controller) style, with each class focusing on a single
aspect:

\begin{aosadescription}

  \item{\emph{Sheet}} is the data model, representing an in-memory
  structure of a spreadsheet.
  It contains a dictionary from
  coordinates to \emph{Cell} objects, each representing a single cell.
  Empty cells need no entries, and hence consume no memory at all.

  \item{\emph{Cell}} represents a cell's content and formats.  Some
  common properties are shown in \aosatblref{tbl.soc.cellcontents}.

  \item{\emph{RenderContext}} implements the view; it is responsible
  for rendering a sheet into DOM objects.

  \item{\emph{TableControl}} is the main controller, accepting mouse
  and keyboard events.
  As it receives view events such as scrolling
  and resizing, it updates its associated \emph{RenderContext}
  object.  As it receives update events that affects the sheet's
  content, it schedules new commands to the sheet's command queue.

  \item{\emph{SpreadSheetControl}} is the top-level UI with
  toolbars, status bars, dialog boxes and color pickers.

  \item{\emph{SpreadSheetViewer}} is an alternate top-level UI that
  provides a read-only interactive view.

\end{aosadescription}

\begin{table}\centering
  \begin{tabular}{|ll|}
    \hline
    \code{datatype} & \code{t} \\
    \code{datavalue} & \code{1Q84} \\
    \code{color} & \code{black} \\
    \code{bgcolor} & \code{white} \\
    \code{font} & \code{italic bold 12pt Ubuntu} \\
    \code{comment} & \code{Ichi-Kyu-Hachi-Yon} \\
    \hline
  \end{tabular}
  \caption{Cell Contents and Formats}
  \label{tbl.soc.cellcontents}
\end{table}

We adopted a minimal class-based object system with simple
composition/delegation, and make no use of inheritance or object
prototypes.  All symbols are placed under the \code{SocialCalc.*}
namespace to avoid naming conflicts.

Each update on the sheet goes through the \code{ScheduleSheetCommands}
method, which takes a command string representing the edit. (Some
common commands are show in \aosatblref{tbl.soc.commands}.)  The
application embedding SocialCalc may define extra commands on their
own, by adding named callbacks into the
\code{SocialCalc.SheetCommandInfo.CmdExtensionCallbacks} object, and
use the \code{startcmdextension} command to invoke them.

\begin{table}[ht]
\begin{minipage}[b]{0.4\linewidth}\centering

\begin{verbatim}
    set     sheet defaultcolor blue
    set     A width 100
    set     A1 value n 42
    set     A2 text t Hello
    set     A3 formula A1*2
    set     A4 empty
    set     A5 bgcolor green
    merge   A1:B2
    unmerge A1
\end{verbatim}

\end{minipage}
\hspace{0.5cm}
\begin{minipage}[b]{0.6\linewidth}
\centering

\begin{verbatim}
    erase   A2
    cut     A3
    paste   A4
    copy    A5
    sort    A1:B9 A up B down
    name    define Foo A1:A5
    name    desc   Foo Used in formulas like SUM(Foo)
    name    delete Foo
    startcmdextension UserDefined args
\end{verbatim}

\end{minipage}
\caption{SocialCalc Commands}
\label{tbl.soc.commands}
\end{table}

\end{aosasect1}

\begin{aosasect1}{Command Run-loop}

To improve responsiveness, SocialCalc performs all recalculation and
DOM updates in the background, so the user can keep making changes to
several cells while the engine catches up on earlier changes in the
command queue.

\aosafigure[300pt]{../images/socialcalc/socialcalc-command-runloop.png}{SocialCalc Command Run-loop}{fig.soc.loop}

When a command is running, the \code{TableEditor} object sets its
\code{busy} flag to true; subsequent commands are then pushed into the
\code{deferredCommands} queue, ensuring a sequential order of
execution.  As the event loop diagram in \aosafigref{fig.soc.loop}
shows, the Sheet object keeps sending \code{StatusCallback} events to
notify the user of the current state of command execution, through
each of the four steps:

\begin{aosadescription}

  \item{\emph{ExecuteCommand}}: Sends \code{cmdstart} upon start, and
   \code{cmdend} when the command finishes execution.  If the command
   changed a cell's value indirectly, enter the \emph{Recalc} step.
   Otherwise, if the command changed the visual appearance of one or
   more on-screen cells, enter the \emph{Render} step.  If neither of
   the above applies (for example with the \code{copy} command), skip
   to the \emph{PositionCalculations} step.

\pagebreak

  \item{\emph{Recalc}} \emph{(as{ }needed)}: Sends \code{calcstart} upon start,
  \code{calcorder} every 100ms when checking the dependency chain of
  cells, \code{calccheckdone} when the check finishes, and
  \code{calcfinished} when all affected cells received their
  re-calculated values.  This step is always followed by the \emph{Render}
  step.

  \item{\emph{Render}} \emph{(as{ }needed)}: Sends \code{schedrender} upon
  start, and \code{renderdone} when the
  \code{{\textless}table{\textgreater}} element is updated with
  formatted cells. This step is always followed by \emph{PositionCalculations}.

  \item{\emph{PositionCalculations}}: Sends \code{schedposcalc} upon
  start, and \code{doneposcalc} after updating the scrollbars, the
  current editable cell cursor, and other visual components of the
  \code{TableEditor}.

\end{aosadescription}

Because all commands are saved as they are executed, we naturally get
an audit log of all operations.  The \code{Sheet.CreateAuditString}
method provides a newline-delimited string as the audit trail, with
each command in a single line.

\code{ExecuteSheetCommand} also creates an undo command for each
command it executes.  For example, if the cell A1 contains ``Foo''
and the user executes \code{set A1 text Bar}, then an undo-command
\code{set A1 text Foo} is pushed to the undo stack.  If the user
clicks Undo, then the undo-command is executed to restore A1 to its
original value.

\end{aosasect1}

\begin{aosasect1}{Table Editor}

Now let's look at the TableEditor layer.  It calculates the on-screen
coordinates of its \code{RenderContext}, and manages
horizontal/vertical scroll bars through two \code{TableControl}
instances.

\aosafigure[225pt]{../images/socialcalc/socialcalc-parts.png}{TableControl Instances Manage Scroll Bars}{fig.soc.parts}

The view layer, handled by the \code{RenderContext} class, also
differs from WikiCalc's design.  Instead of mapping each cell to a
\code{{\textless}td{\textgreater}} element, we now simply create a
fixed-size \code{{\textless}table{\textgreater}} that fits the
browser's visible area, and pre-populate it with
\code{{\textless}td{\textgreater}} elements.

As the user scrolls the spreadsheet through our custom-drawn scroll
bars, we dynamically update the \code{innerHTML} of the pre-drawn
\code{{\textless}td{\textgreater}} elements.  This means we don't need to
create or destroy any \code{{\textless}tr{\textgreater}} or
\code{{\textless}td{\textgreater}} elements in many common cases,
which greatly speeds up response time.

Because \code{RenderContext} only renders the visible region, the size
of Sheet object can be arbitrarily large without affecting its
performance.

\code{TableEditor} also contains a \code{CellHandles} object, which
implements the radial fill/move/slide menu attached to the
bottom-right corner to the current editable cell, known as the ECell,
shown in \aosafigref{fig.soc.ecell}.

\aosafigure[180pt]{../images/socialcalc/socialcalc-cell-handles.png}{Current Editable Cell, Known as the ECell}{fig.soc.ecell}

The input box is managed by two classes: \code{InputBox} and
\code{InputEcho}.  The former manages the above-the-grid edit row,
while the latter shows an updated-as-you-type preview layer,
overlaying the ECell's content (\aosafigref{fig.soc.input}).

\aosafigure[200pt]{../images/socialcalc/socialcalc-input.png}{The Input Box is Managed by Two Classes}{fig.soc.input}

Usually, the SocialCalc engine only needs to communicate to the server
when opening a spreadsheet for edit, and when saving it back to
server.  For this purpose, the \code{Sheet.ParseSheetSave} method
parses a save format string into a \code{Sheet} object, and the
\code{Sheet.CreateSheetSave} method serializes a \code{Sheet} object
back into the save format.

Formulas may refer to values from any remote spreadsheet with a URL.
The \code{recalc} command re-fetches the externally referenced
spreadsheets, parses them again with \code{Sheet.ParseSheetSave}, and
stores them in a cache so the user can refer to other cells in the
same remote spreadsheets without re-fetching its content.

\end{aosasect1}

\begin{aosasect1}{Save Format}

The save format is in standard MIME \code{multipart/mixed} format,
consisting of four \code{text/plain; charset=UTF-8} parts, each part
containing newline-delimited text with colon-delimited data fields.
The parts are:

\begin{aosaitemize}

  \item The \code{meta} part lists the types of the other parts.

  \item The \code{sheet} part lists each cell's format and content, each
  column's width (if not default), the sheet's default format, followed
  by a list of fonts, colors and borders used in the sheet.

  \item The optional \code{edit} part saves the \code{TableEditor}'s
  edit state, including ECell's last position, as well as the fixed sizes of
  row/column panes.

  \item The optional \code{audit} part contains the history of
 commands executed in the previous editing session.

\end{aosaitemize}

For example, \aosafigref{fig.soc.save} shows a spreadsheet with three
cells, with \code{1874} in A1 as the ECell, the formula \code{2\^{}2*43}
in A2, and the formula \code{SUM(Foo)} in A3 rendered in bold,
referring to the named range \code{Foo} over \code{A1:A2}.

\aosafigure[180pt]{../images/socialcalc/socialcalc-2046.png}{A Spreadsheet with Three Cells}{fig.soc.save}

The serialized save format for the spreadsheet looks like this:

\begin{verbatim}
    socialcalc:version:1.0
    MIME-Version: 1.0
    Content-Type: multipart/mixed; boundary=SocialCalcSpreadsheetControlSave
    --SocialCalcSpreadsheetControlSave
    Content-type: text/plain; charset=UTF-8

    # SocialCalc Spreadsheet Control Save
    version:1.0
    part:sheet
    part:edit
    part:audit
    --SocialCalcSpreadsheetControlSave
    Content-type: text/plain; charset=UTF-8

    version:1.5
    cell:A1:v:1874
    cell:A2:vtf:n:172:2^2*43
    cell:A3:vtf:n:2046:SUM(Foo):f:1
    sheet:c:1:r:3
    font:1:normal bold * *
    name:FOO::A1\cA2
    --SocialCalcSpreadsheetControlSave
    Content-type: text/plain; charset=UTF-8

    version:1.0
    rowpane:0:1:14
    colpane:0:1:16
    ecell:A1
    --SocialCalcSpreadsheetControlSave
    Content-type: text/plain; charset=UTF-8

    set A1 value n 1874
    set A2 formula 2^2*43
    name define Foo A1:A2
    set A3 formula SUM(Foo)
    --SocialCalcSpreadsheetControlSave--
\end{verbatim}

\noindent This format is designed to be human-readable, as well as being
relatively easy to generate programmatically.  This makes it possible
for Drupal's Sheetnode plugin to use PHP to convert
between this format and other popular spreadsheet formats, such as
Excel (\code{.xls}) and OpenDocument (\code{.ods}).

Now that we have a good idea about how the pieces in SocialCalc fit
together, let's look at two real-world examples of extending
SocialCalc.

\end{aosasect1}

\begin{aosasect1}{Rich-text Editing}

The first example we'll look at is enhancing SocialCalc's text cells
with wiki markup to display its rich-text rendering right in the
table editor (\aosafigref{fig.soc.rt}).

\aosafigure[200pt]{../images/socialcalc/richtext-screenshot.png}{Rich Text Rendering in the Table Editor}{fig.soc.rt}

We added this feature to SocialCalc right after its 1.0 release, to
address the popular request of inserting images, links and text
markups using a unified syntax.  Since Socialtext already has an
open-source wiki platform, it was natural to re-use the syntax for
SocialCalc as well.

To implement this, we need a custom renderer for the
\code{textvalueformat} of \code{text-wiki}, and to change the default
format for text cells to use it.

What is this \code{textvalueformat}, you ask?  Read on.

\begin{aosasect2}{Types and Formats}

In SocialCalc, each cell has a \code{datatype} and a \code{valuetype}.
Data cells with text or numbers correspond to text/numeric value
types, and formula cells with \code{datatype="f"} may generate either
numeric or text values.

Recall that on the Render step, the \code{Sheet} object generates HTML
from each of its cells.  It does so by inspecting each cell's
\code{valuetype}: If it begins with t, then the cell's
\code{textvalueformat} attribute determines how generation is done.
If it begins with \code{n}, then the \code{nontextvalueformat} attribute is
used instead.

However, if the cell's \code{textvalueformat} or
\code{nontextvalueformat} attribute is not defined explicitly, then a
default format is looked up from its \code{valuetype}, as shown in
\aosafigref{fig.soc.vformat}.

\aosafigure[300pt]{../images/socialcalc/richtext-formats.png}{Value Types}{fig.soc.vformat}

% \pagebreak

\noindent Support for the \code{text-wiki} value format is coded in
\code{SocialCalc.format\_text\_for\_display}:

\begin{verbatim}
if (SocialCalc.Callbacks.expand_wiki && /^text-wiki/.test(valueformat)) {
    // do general wiki markup
    displayvalue = SocialCalc.Callbacks.expand_wiki(
        displayvalue, sheetobj, linkstyle, valueformat
    );
}
\end{verbatim}

Instead of inlining the wiki-to-HTML expander in
\code{format\_text\_for\_display}, we will define a new hook in
\code{SocialCalc.Callbacks}.  This is the recommended style
throughout the SocialCalc codebase; it improves modularity by making
it possible to plug in different ways of expanding wikitext, as well
as keeping compatibility with embedders that do not desire this
feature.

\end{aosasect2}

\begin{aosasect2}{Rendering Wikitext}

Next, we'll make use of
Wikiwyg\footnote{\url{https://github.com/audreyt/wikiwyg-js}}, a
Javascript library offering two-way conversions between wikitext and
HTML.

We define the \code{expand\_wiki} function by taking the cell's text,
running it through Wikiwyg's wikitext parser and its HTML emitter:

\begin{verbatim}
var parser = new Document.Parser.Wikitext();
var emitter = new Document.Emitter.HTML();
SocialCalc.Callbacks.expand_wiki = function(val) {
    // Convert val from Wikitext to HTML
    return parser.parse(val, emitter);
}
\end{verbatim}

\pagebreak 
\noindent The final step involves scheduling the \code{set sheet
defaulttextvalueformat text-wiki} command right after the
spreadsheet initializes:


\begin{verbatim}
// We assume there's a <div id="tableeditor"/> in the DOM already
var spreadsheet = new SocialCalc.SpreadsheetControl();
spreadsheet.InitializeSpreadsheetControl("tableeditor", 0, 0, 0);
spreadsheet.ExecuteCommand('set sheet defaulttextvalueformat text-wiki');
\end{verbatim}

\noindent Taken together, the Render step now works as shown in
\aosafigref{fig.soc.render}.

\aosafigure[349pt]{../images/socialcalc/richtext-flow.png}{Render Step}{fig.soc.render}

That's all!  The enhanced SocialCalc now supports a rich set of wiki
markup syntax:

\begin{verbatim}
*bold* _italic_ `monospace` {{unformatted}}
> indented text
* unordered list
# ordered list
"Hyperlink with label"<http://softwaregarden.com/>
{image: http://www.socialtext.com/images/logo.png}
\end{verbatim}

\noindent Try entering \code{*bold* \_italic\_ `monospace`} in A1, and you'll
see it rendered as rich text (\aosafigref{fig.soc.rtext}).

\aosafigure[150pt]{../images/socialcalc/richtext-example.png}{Wikywyg Example}{fig.soc.rtext}

\end{aosasect2}

\end{aosasect1}

% \pagebreak

\begin{aosasect1}{Real-time Collaboration}

The next example we'll explore is multi-user, real-time editing on a
shared spreadsheet.  This may seem complicated at first, but thanks to
SocialCalc's modular design all it takes is for each on-line user to
broadcast their commands to other participants.

To distinguish between locally-issued commands and remote commands, we
add an \code{isRemote} parameter to the \code{ScheduleSheetCommands}
method:

\begin{verbatim}
SocialCalc.ScheduleSheetCommands = function(sheet, cmdstr, saveundo, isRemote) {
   if (SocialCalc.Callbacks.broadcast && !isRemote) {
       SocialCalc.Callbacks.broadcast('execute', {
           cmdstr: cmdstr, saveundo: saveundo
       });
   }
   // ...original ScheduleSheetCommands code here...
}
\end{verbatim}

\noindent Now all we need to do is to define a suitable
\code{SocialCalc.Callbacks.broadcast} callback function.  Once it's
in place, the same commands will be executed on all users connected
to the same spreadsheet.

When this feature was first implemented for OLPC (One Laptop Per 
Child\footnote{\url{http://one.laptop.org/}}) by SEETA's Sugar 
Labs\footnote{\url{http://seeta.in/wiki/index.php?title=Collaboration_in_SocialCalc}}
in 2009, the \code{broadcast} function was built with XPCOM calls into
D-Bus/Telepathy, the standard transport for OLPC/Sugar networks (see
\aosafigref{fig.soc.olpc}).

\aosafigure[280pt]{../images/socialcalc/collab-olpc.png}{OLPC Implementation}{fig.soc.olpc}

That worked reasonably well, enabling XO instances in the same Sugar
network to collaborate on a common SocialCalc spreadsheet.  However,
it is both specific to the Mozilla/XPCOM browser platform, as well as
to the D-Bus/Telepathy messaging platform.

\begin{aosasect2}{Cross-browser Transport}

To make this work across browsers and operating systems, we use the
\code{Web::Hippie}\footnote{\url{http://search.cpan.org/dist/Web-Hippie/}}
framework, a high-level abstraction of JSON-over-WebSocket with
convenient jQuery bindings, with MXHR (Multipart XML HTTP
Request\footnote{\url{http://about.digg.com/blog/duistream-and-mxhr}})
as the fallback transport mechanism if WebSocket is not available.

For browsers with Adobe Flash plugin installed but without native
WebSocket support, we use the
\code{web\_socket.js}\footnote{\url{https://github.com/gimite/web-socket-js}}
project's Flash emulation of WebSocket, which is often faster and more reliable
than MXHR.  The operation flow is shown in \aosafigref{fig.soc.collab}.

\aosafigure{../images/socialcalc/collab-flow.png}{Cross-Browser Flow}{fig.soc.collab}

\pagebreak

The client-side \code{SocialCalc.Callbacks.broadcast} function is
defined as:

\begin{verbatim}
var hpipe = new Hippie.Pipe();

SocialCalc.Callbacks.broadcast = function(type, data) {
    hpipe.send({ type: type, data: data });
};

$(hpipe).bind("message.execute", function (e, d) {
    var sheet = SocialCalc.CurrentSpreadsheetControlObject.context.sheetobj;
    sheet.ScheduleSheetCommands(
        d.data.cmdstr, d.data.saveundo, true // isRemote = true
    );
    break;
});
\end{verbatim}

\noindent Although this works quite well, there are still two remaining issues
to resolve.

\end{aosasect2}

\begin{aosasect2}{Conflict Resolution}

The first one is a race-condition in the order of commands executed:
If users A and B simultaneously perform an operation affecting the
same cells, then receive and execute commands broadcast from the other
user, they will end up in different states, as shown in
\aosafigref{fig.soc.conflict}.

\aosafigure{../images/socialcalc/collab-conflict.png}{Race Condition Conflict}{fig.soc.conflict}

We can resolve this with SocialCalc's built-in undo/redo mechanism, as
shown in \aosafigref{fig.soc.resolve}.

\aosafigure{../images/socialcalc/collab-resolution.png}{Race Condition Conflict Resolution}{fig.soc.resolve}

The process used to resolve the conflict is as follows.  When a client
broadcasts a command, it adds the command to a Pending queue.  When a client
receives a command, it checks the remote command against the Pending queue.

If the Pending queue is empty, then the command is simply executed as a remote
action. If the remote command matches a command in the Pending queue, then the
local command is removed from the queue.

Otherwise, the client checks if there are any queued commands that conflict
with the received command.  If there are conflicting commands, the client first
\code{Undo}es those commands and marks them for later \code{Redo}.  After
undoing the conflicting commands (if any), the remote command is executed as
usual.

When a marked-for-redo command is received from the server, the client will
execute it again, then remove it from the queue.


\end{aosasect2}

\begin{aosasect2}{Remote Cursors}

Even with race conditions resolved, it is still suboptimal to
accidentally overwrite the cell another user is currently editing.  A
simple improvement is for each client to broadcast its cursor position
to other users, so everyone can see which cells are being worked on.

To implement this idea, we add another \code{broadcast} handler to the
\code{MoveECellCallback} event:

\begin{verbatim}
editor.MoveECellCallback.broadcast = function(e) {
    hpipe.send({
        type: 'ecell',
        data: e.ecell.coord
    });
};

$(hpipe).bind("message.ecell", function (e, d) {
    var cr = SocialCalc.coordToCr(d.data);
    var cell = SocialCalc.GetEditorCellElement(editor, cr.row, cr.col);
    // ...decorate cell with styles specific to the remote user(s) on it...
});
\end{verbatim}

To mark cell focus in spreadsheets, it's common to use colored
borders.  However, a cell may already define its own \code{border}
property, and since \code{border} is mono-colored, it can only
represent one cursor on the same cell.

Therefore, on browsers with support for CSS3, we use the \code{box-shadow}
property to represent multiple peer cursors in the same cell:

\begin{verbatim}
/* Two cursors on the same cell */
box-shadow: inset 0 0 0 4px red, inset 0 0 0 2px green;
\end{verbatim}

\aosafigref{fig.soc.borders} shows how the screen would look with four
people editing on the same spreadsheet.

\aosafigure[150pt]{../images/socialcalc/collab-borders.png}{Four Users Editing One Spreadsheet}{fig.soc.borders}

\end{aosasect2}

\end{aosasect1}

\begin{aosasect1}{Lessons Learned}

We delivered SocialCalc 1.0 on October 19th, 2009, the 30th
anniversary of the initial release of VisiCalc.  The experience of
collaborating with my colleagues at Socialtext under Dan Bricklin's
guidance was very valuable to me, and I'd like to share some lessons
I learned during that time.

\begin{aosasect2}{Chief Designer with a Clear Vision}

In \cite{bib:brooks:design}, Fred Brooks argues that when building
complex systems, the conversation is much more direct if we focus on a
coherent \emph{design concept}, rather than derivative
representations. According to Brooks, the formulation of such a
coherent design concept is best kept in a single person's mind:

\begin{quotation}

  \noindent
  Since conceptual integrity is the most important attribute of a
  great design, and since that comes from one or a few minds working
  \emph{uno animo}, the wise manager boldly entrusts each design task to a
  gifted chief designer.

\end{quotation}

In the case of SocialCalc, having Tracy Ruggles as our chief
user-experience designer was the key for the project to converge
toward a shared vision.  Since the underlying SocialCalc engine was
so malleable, the temptation of feature creep was very real. Tracy's
ability to communicate using design sketches really helped us
present features in a way that feels intuitive to users.

\end{aosasect2}

\begin{aosasect2}{Wikis for Project Continuity}

Before I joined the SocialCalc project, there was already over two
years' worth of ongoing design and development, but I was able to
catch up and start contributing in less than a week, simply due to
the fact that \emph{everything is in the wiki}. From the earliest
design notes to the most up-to-date browser support matrix, the
entire process was chronicled in wiki pages and SocialCalc
spreadsheets.

Reading through the project's workspace brought me quickly to the same
page as others, without the usual hand-holding overhead
typically associated with orienting a new team member.

This would not be possible in traditional open source projects, where
most conversation takes place on IRC and mailing lists and the wiki
(if present) is only used for documentations and links to development
resources.  For a newcomer, it's much more difficult to reconstruct
context from unstructured IRC logs and mail archives.

\end{aosasect2}

\begin{aosasect2}{Embrace Time Zone Differences}

David Heinemeier Hansson, creator of Ruby on Rails, once remarked on
the benefit of distributed teams when he first joined 37signals. "The
seven time zones between Copenhagen and Chicago actually meant that
we got a lot done with few interruptions." With nine time zones
between Taipei and Palo Alto, that was true for us during
SocialCalc's development as well.

We often completed an entire Design-Development-QA feedback cycle
within a 24-hour day, with each aspect taking one person's 8-hour
work day in their local daytime.  This asynchronous style of
collaboration compelled us to produce self-descriptive artifacts
(design sketch, code and tests), which in turn greatly improved our
trust in each other.

\end{aosasect2}

\begin{aosasect2}{Optimize for Fun}

In my 2006 keynote for the CONISLI conference~\cite{bib:tang:fun}, I
summarized my experience leading a distributed team implementing the
Perl 6 language into a few observations.  Among them, \emph{Always
 have a Roadmap}, \emph{Forgiveness {\textgreater} Permission},
\emph{Remove deadlocks}, \emph{Seek ideas, not consensus}, and
\emph{Sketch ideas with code} are particularly relevant for small
distributed teams.

When developing SocialCalc, we took great care in distributing
knowledge among team members with collaborative code ownership, so
nobody would become a critical bottleneck.

Furthermore, we pre-emptively resolved disputes by actually coding up
alternatives to explore the design space, and were not afraid of
replacing fully-working prototypes when a better design arrived.

These cultural traits helped us foster a sense of anticipation and
camaraderie despite the absence of face-to-face interaction, kept
politics to a minimum, and made working on SocialCalc a lot of fun.

\end{aosasect2}

\begin{aosasect2}{Drive Development with Story Tests}

Prior to joining Socialtext, I've advocated the ``interleave tests
with the specification'' approach, as can be seen in the Perl 6
specification\footnote{\url{http://perlcabal.org/syn/S02.html}}, where
we annotate the language specification with the official test suite.
However, it was Ken Pier and Matt Heusser, the QA team for SocialCalc,
who really opened my eyes to how this can be taken to the next level,
bringing tests to the place of \emph{executable specification}.

% \pagebreak

In Chapter 16 of \cite{bib:goucher:test}, Matt explained our
story-test driven development process as follows:

\begin{quotation}

  The basic unit of work is a "story," which is an extremely
  lightweight requirements document. A story contains a brief
  description of a feature along with examples of what needs to happen
  to consider the story completed; we call these examples "acceptance
  tests" and describe them in plain English.

  During the initial cut of the story, the product owner makes a
  good-faith first attempt to create acceptance tests, which are
  augmented by developers and testers before any developer writes
  a line of code.

\end{quotation}

These story tests are then translated into wikitests, a table-based
specification language inspired by Ward Cunningham's FIT
framework\footnote{\url{http://fit.c2.com/}}, which drives automated
testing frameworks such as
\code{Test::WWW::Mechanize}\footnote{\url{http://search.cpan.org/dist/Test-WWW-Mechanize/}}
and
\code{Test::WWW::Selenium}\footnote{\url{http://search.cpan.org/dist/Test-WWW-Selenium/}}.

It's hard to overstate the benefit of having story tests as a common
language to express and validate requirements. It was instrumental in
reducing misunderstanding, and has all but eliminated regressions from
our monthly releases.

\end{aosasect2}

% \pagebreak

\begin{aosasect2}{Open Source With CPAL}

Last but not least, the open source model we chose for SocialCalc
makes an interesting lesson in itself.

% \vspace*{1.4ex}

Socialtext created the Common Public Attribution
License\footnote{\url{https://www.socialtext.net/open/?cpal}} for
SocialCalc.  Based on the Mozilla Public License, CPAL is designed to
allow the original author to require an attribution to be displayed
on the software's user interface, and has a network-use clause that
triggers share-alike provisions when derived work is hosted by a
service over the network.

% \vspace*{1.4ex}

After its approval by both the Open Source
Initiative\footnote{\url{http://opensource.org/}} and the Free
Software Foundation\footnote{\url{http://www.fsf.org}}, we've seen prominent
sites such as Facebook\footnote{\url{https://github.com/facebook/platform}}
and Reddit\footnote{\url{https://github.com/reddit/reddit}} opting to release
their platform's source code under the CPAL, which is very
encouraging.

% \vspace*{1.4ex}

Because CPAL is a ``weak copyleft'' license, developers can freely
combine it with either free or proprietary software, and only need to
release modifications to SocialCalc itself.  This enabled various
communities to adopt SocialCalc and made it more awesome.

% \vspace*{1.4ex}

There are many interesting possibilities with this open-source
spreadsheet engine, and if you can find a way to embed SocialCalc
into your favorite project, we'd definitely love to hear about it.

\end{aosasect2}

\end{aosasect1}

\end{aosachapter}
