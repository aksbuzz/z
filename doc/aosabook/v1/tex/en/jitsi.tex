\begin{aosachapter}{Jitsi}{s:jitsi}{Emil Ivov}

Jitsi is an application that allows people to make video and voice
calls, share their desktops, and exchange files and messages. More
importantly it allows people to do this over a number of different
protocols, ranging from the standardized XMPP (Extensible Messaging
and Presence Protocol) and SIP (Session Initiation Protocol) to
proprietary ones like Yahoo! and Windows Live Messenger (MSN).  It
runs on Microsoft Windows, Apple Mac OS X, Linux, and FreeBSD\@. It is
written mostly in Java but it also contains parts written in native
code.  In this chapter, we'll look at Jitsi's OSGi-based architecture,
see how it implements and manages protocols, and look back on what
we've learned from building it.\footnote{To refer directly to the
source as you read, download it from
\url{http://jitsi.org/source}.  If you are using Eclipse or NetBeans,
you can go to \url{http://jitsi.org/eclipse} or
\url{http://jitsi.org/netbeans} for instructions on how configure
them.}

\begin{aosasect1}{Designing Jitsi}

The three most important constraints that we had to keep in mind when
designing Jitsi (at the time called SIP Communicator) were
multi-protocol support, cross-platform operation, and
developer-friendliness.

From a developer's perspective, being multi-protocol comes down to
having a common interface for all protocols. In other words, when a
user sends a message, our graphical user interface needs to always
call the same \code{sendMessage} method regardless of whether the
currently selected protocol actually uses a method called
\code{sendXmppMessage} or \code{sendSipMsg}.

The fact that most of our code is written in Java satisfies, to a
large degree, our second constraint: cross-platform operation. Still,
there are things that the Java Runtime Environment (JRE) does not
support or does not do the way we'd like it to, such as capturing
video from your webcam. Therefore, we need to use DirectShow on
Windows, QTKit on Mac OS X, and Video for Linux 2 on Linux. Just as
with protocols, the parts of the code that control video calls cannot
be bothered with these details (they are complicated enough as it is).

Finally, being developer-friendly means that it should be easy for
people to add new features. There are millions of people using VoIP
today in thousands of different ways; various service providers and
server vendors come up with different use cases and ideas about new
features. We have to make sure that it is easy for them to use Jitsi
the way they want.  Someone who needs to add something new should have
to read and understand only those parts of the project they are
modifying or extending.  Similarly, one person's changes should have
as little impact as possible on everyone else's work.

To sum up, we needed an environment where different parts of the code
are relatively independent from each other. It had to be possible to
easily replace some parts depending on the operating system; have
others, like protocols, run in parallel and yet act the same; and it
had to be possible to completely rewrite any one of those parts and
have the rest of the code work without any changes.  Finally, we
wanted the ability to easily switch parts on and off, as well as the
ability to download plugins over the Internet to our list.

We briefly considered writing our own framework, but soon dropped the
idea. We were itching to start writing VoIP and IM code as soon as
possible, and spending a couple of months on a plugin framework
didn't seem that exciting. Someone suggested OSGi, and it seemed to be
the perfect fit.

\end{aosasect1}

\begin{aosasect1}{Jitsi and the OSGi Framework}

People have written entire books about OSGi, so we're not going to go
over everything the framework stands for. Instead we will only explain
what it gives us and the way we use it in Jitsi.

Above everything else, OSGi is about modules.  Features in OSGi
applications are separated into bundles. An OSGi bundle is little more
than a regular JAR file like the ones used to distribute Java
libraries and applications. Jitsi is a collection of such
bundles. There is one responsible for connecting to Windows Live
Messenger, another one that does XMPP, yet another one that handles
the GUI, and so on.  All these bundles run together in an environment
provided, in our case, by Apache Felix, an open source OSGi
implementation.

All these modules need to work together. The GUI bundle needs to send
messages via the protocol bundles, which in turn need to store them
via the bundles handling message history. This is what OSGi services
are for: they represent the part of a bundle that is visible to
everyone else. An OSGi service is most often a group of Java
interfaces that allow use of a specific functionality like logging,
sending messages over the network, or retrieving the list of recent
calls. The classes that actually implement the functionality are known
as a service implementation. Most of them carry the name of the
service interface they implement, with an ``Impl'' suffix at the end
(e.g., \code{ConfigurationServiceImpl}). The OSGi framework allows
developers to hide service implementations and make sure that they are
never visible outside the bundle they are in. This way, other bundles
can only use them through the service interfaces.

Most bundles also have activators. Activators are simple interfaces
that define a \code{start} and a \code{stop} method. Every time Felix
loads or removes a bundle in Jitsi, it calls these methods so that the
bundle can prepare to run or shut down. When calling these methods
Felix passes them a parameter called BundleContext. The BundleContext
gives bundles a way to connect to the OSGi environment. This way they
can discover whatever OSGi service they need to use, or register one
themselves (\aosafigref{fig.jit.osgi}).

\aosafigure{../images/jitsi/OSGI.png}{OSGi Bundle Activation}{fig.jit.osgi}

So let's see how this actually works.  Imagine a service that
persistently stores and retrieves properties. In Jitsi this is what we
call the ConfigurationService and it looks like this:

\pagebreak 

\begin{verbatim}
package net.java.sip.communicator.service.configuration;

public interface ConfigurationService
{
  public void setProperty(String propertyName, Object property);
  public Object getProperty(String propertyName);
}
\end{verbatim}

A very simple implementation of the \code{ConfigurationService} looks like
this:

\begin{verbatim}
package net.java.sip.communicator.impl.configuration;

import java.util.*;
import net.java.sip.communicator.service.configuration.*;

public class ConfigurationServiceImpl implements ConfigurationService
{
  private final Properties properties = new Properties();

  public Object getProperty(String name)
  {
    return properties.get(name);
  }

  public void setProperty(String name, Object value)
  {
    properties.setProperty(name, value.toString());
  }
}
\end{verbatim}

\noindent Notice how the service is defined in the
\code{net.java.sip.communicator.\emph{service}} package, while the
implementation is in \code{net.java.sip.communicator.\emph{impl}}. All
services and implementations in Jitsi are separated under these two
packages. OSGi allows bundles to only make some packages visible
outside their own JAR, so the separation makes it easier for bundles
to only \emph{export} their service packages and keep their
implementations hidden.

The last thing we need to do so that people can start using our
implementation is to register it in the \code{BundleContext} and
indicate that it provides an implementation of the
\code{ConfigurationService}. Here's how this happens:

\begin{verbatim}
package net.java.sip.communicator.impl.configuration;

import org.osgi.framework.*;
import net.java.sip.communicator.service.configuration;

public class ConfigActivator implements BundleActivator
{
  public void start(BundleContext bc) throws Exception
  {
    bc.registerService(ConfigurationService.class.getName(), // service name
         new ConfigurationServiceImpl(), // service implementation
         null);
  }
}
\end{verbatim}

\noindent Once the \code{ConfigurationServiceImpl} class is registered in the
\code{BundleContext}, other bundles can start using it. Here's an
example showing how some random bundle can use our configuration
service:

\begin{verbatim}
package net.java.sip.communicator.plugin.randombundle;

import org.osgi.framework.*;
import net.java.sip.communicator.service.configuration.*;

public class RandomBundleActivator implements BundleActivator
{
  public void start(BundleContext bc) throws Exception
  {
    ServiceReference cRef = bc.getServiceReference(
                              ConfigurationService.class.getName());
    configService = (ConfigurationService) bc.getService(cRef);

    // And that's all! We have a reference to the service implementation
    // and we are ready to start saving properties:
    configService.setProperty("propertyName", "propertyValue");
  }
}
\end{verbatim}

\noindent Once again, notice the package. In
\code{net.java.sip.communicator.plugin} we keep bundles that use
services defined by others but that neither export nor implement any
themselves. Configuration forms are a good example of such plugins:
They are additions to the Jitsi user interface that allow users to
configure certain aspects of the application. When users change
preferences, configuration forms interact with the
\code{ConfigurationService} or directly with the bundles responsible
for a feature. However, none of the other bundles ever need to
interact with them in any way (\aosafigref{fig.jit.pkgs}).

\aosafigure[250pt]{../images/jitsi/PKGs.png}{Service Structure}{fig.jit.pkgs}

\end{aosasect1}

\begin{aosasect1}{Building and Running a Bundle}

Now that we've seen how to write the code in a bundle, it's time to
talk about packaging. When running, all bundles need to indicate three
different things to the OSGi environment: the Java packages they make
available to others (i.e. exported packages), the ones that they would
like to use from others (i.e. imported packages), and the name of
their BundleActivator class. Bundles do this through the manifest of
the JAR file that they will be deployed in.

For the \code{ConfigurationService} that we defined above, the
manifest file could look like this:

\begin{verbatim}
Bundle-Activator: net.java.sip.communicator.impl.configuration.ConfigActivator
Bundle-Name: Configuration Service Implementation
Bundle-Description: A bundle that offers configuration utilities
Bundle-Vendor: jitsi.org
Bundle-Version: 0.0.1
System-Bundle: yes
Import-Package: org.osgi.framework,
Export-Package: net.java.sip.communicator.service.configuration
\end{verbatim}

After creating the JAR manifest, we are ready to create the bundle
itself. In Jitsi we use Apache Ant to handle all build-related
tasks. In order to add a bundle to the Jitsi build process, you need
to edit the \code{build.xml} file in the root directory of the
project.  Bundle JARs are created at the bottom of the
\code{build.xml} file, with \code{bundle-xxx} targets. In order to
build our configuration service we need the following:

\begin{verbatim}
<target name="bundle-configuration">
  <jar destfile="${bundles.dest}/configuration.jar" manifest=
    "${src}/net/java/sip/communicator/impl/configuration/conf.manifest.mf" >

    <zipfileset dir="${dest}/net/java/sip/communicator/service/configuration"
        prefix="net/java/sip/communicator/service/configuration"/>
    <zipfileset dir="${dest}/net/java/sip/communicator/impl/configuration"
        prefix="net/java/sip/communicator/impl/configuration" />
  </jar>
</target>
\end{verbatim}

As you can see, the Ant target simply creates a JAR file using our
configuration manifest, and adds to it the configuration packages from
the \code{service} and \code{impl} hierarchies. Now the only thing
that we need to do is to make Felix load it.

We already mentioned that Jitsi is merely a collection of OSGi
bundles. When a user executes the application, they actually start
Felix with a list of bundles that it needs to load.  You can find that
list in our \code{lib} directory, inside a file called
\code{felix.client.run.properties}. Felix starts bundles in the order
defined by start levels: All those within a particular level are
guaranteed to complete before bundles in subsequent levels start
loading. Although you can't see this in the example code above, our
configuration service stores properties in files so it needs to use
our \emph{FileAccessService}, shipped within the \code{fileaccess.jar}
file. We'll therefore make sure that the ConfigurationService starts
after the FileAccessService:

\begin{verbatim}
...
felix.auto.start.30= \
  reference:file:sc-bundles/fileaccess.jar

felix.auto.start.40= \
  reference:file:sc-bundles/configuration.jar \
  reference:file:sc-bundles/jmdnslib.jar \
  reference:file:sc-bundles/provdisc.jar \
...
\end{verbatim}

If you look at the \code{felix.client.run.properties} file, you'll see
a list of packages at the beginning:

\begin{verbatim}
org.osgi.framework.system.packages.extra= \
  apple.awt; \
  com.apple.cocoa.application; \
  com.apple.cocoa.foundation; \
  com.apple.eawt; \
...
\end{verbatim}

\noindent The list tells Felix what packages it needs to make available to
bundles from the system classpath. This means that packages that are
on this list can be imported by bundles (i.e. added to their
\emph{Import-Package} manifest header) without any being exported by
any other bundle. The list mostly contains packages that come from
OS-specific JRE parts, and Jitsi developers rarely need to add new
ones to it; in most cases packages are made available by bundles.

\end{aosasect1}

\begin{aosasect1}{Protocol Provider Service}

The \code{ProtocolProviderService} in Jitsi defines the way all
protocol implementations behave. It is the interface that other
bundles (like the user interface) use when they need to send and
receive messages, make calls, and share files through the networks
that Jitsi connects to.

The protocol service interfaces can all be found under the
\path{net.java.sip.communicator.service.protocol} package.  There are
multiple implementations of the service, one per supported protocol,
and all are stored in
\code{net.java.sip.communicator.impl.protocol.\-protocol\_name}.

Let's start with the \code{service.protocol} directory. The most
prominent piece is the \emph{ProtocolProviderService} interface.
Whenever someone needs to perform a protocol-related task, they have
to look up an implementation of that service in the
\code{BundleContext}. The service and its implementations allow Jitsi
to connect to any of the supported networks, to retrieve the connection
status and details, and most importantly to obtain references to the
classes that implement the actual communications tasks like chatting
and making calls.

\begin{aosasect2}{Operation Sets}

As we mentioned earlier, the \code{ProtocolProviderService} needs to
leverage the various communication protocols and their
differences. While this is particularly simple for features that all
protocols share, like sending a message, things get trickier for tasks
that only some protocols support. Sometimes these differences come
from the service itself: For example, most of the SIP services out
there do not support server-stored contact lists, while this is a
relatively well-supported feature with all other protocols. MSN and
AIM are another good example: at one time neither of them offered the
ability to send messages to offline users, while everyone else
did. (This has since changed.)

The bottom line is our \code{ProtocolProviderService} needs to have a
way of handling these differences so that other bundles, like the GUI,
act accordingly; there's no point in adding a call button to an AIM
contact if there's no way to actually make a call.

OperationSets to the rescue
(\aosafigref{fig.jit.ops}). Unsurprisingly, they are sets of
operations, and provide the interface that Jitsi bundles use to
control the protocol implementations. The methods that you find in an
operation set interface are all related to a particular feature.
\emph{OperationSetBasicInstantMessaging}, for instance, contains
methods for creating and sending instant messages, and registering
listeners that allow Jitsi to retrieve messages it receives. Another
example, \emph{OperationSetPresence}, has methods for querying the
status of the contacts on your list and setting a status for
yourself. So when the GUI updates the status it shows for a contact,
or sends a message to a contact, it is first able to ask the
corresponding provider whether they support presence and
messaging. The methods that \code{ProtocolProviderService} defines for
that purpose are:

\begin{verbatim}
public Map<String, OperationSet> getSupportedOperationSets();
public <T extends OperationSet> T getOperationSet(Class<T> opsetClass);
\end{verbatim}

OperationSets have to be designed so that it is unlikely that a new
protocol we add has support for only some of the operations defined in
an OperationSet. For example, some protocols do not support server-stored
contact lists even though they allow users to query each other's status. 
Therefore, rather than combining the presence management and buddy list 
retrieval features in \code{OperationSetPresence}, we also defined an
\code{OperationSetPersistentPresence} which is only used with protocols
that can store contacts online. On the other hand, we have yet to come 
across a protocol that only allows sending messages without receiving 
any, which is why things like sending and receiving messages can be 
safely combined.

\aosafigureTop{../images/jitsi/OperationSets.png}{Operation Sets}{fig.jit.ops}

\end{aosasect2}

\begin{aosasect2}{Accounts, Factories and Provider Instances}

An important characteristic of the \code{ProtocolProviderService} is
that one instance corresponds to one protocol account. Therefore, at
any given time you have as many service implementations in the
\code{BundleContext} as you have accounts registered by the user.

At this point you may be wondering who creates and registers the
protocol providers.  There are two different entities involved. First,
there is \code{ProtocolProviderFactory}. This is the service that
allows other bundles to instantiate providers and then registers them
as services. There is one factory per protocol and every factory is
responsible for creating providers for that particular
protocol. Factory implementations are stored with the rest of the
protocol internals. For SIP, for example we have
\code{net.java.sip.communicator.impl.protocol.sip.ProtocolProviderFactorySipImpl}.

The second entity involved in account creation is the protocol wizard.
Unlike factories, wizards are separated from the rest of the protocol
implementation because they involve the graphical user interface. The
wizard that allows users to create SIP accounts, for example, can be
found in \code{net.java.sip.communicator.plugin.sipaccregwizz}.

\end{aosasect2}

\end{aosasect1}

\begin{aosasect1}{Media Service}

When working with real-time communication over IP, there is one
important thing to understand: protocols like SIP and XMPP, while
recognized by many as the most common VoIP protocols, are not the ones
that actually move voice and video over the Internet. This task is
handled by the Real-time Transport Protocol (RTP).  SIP and XMPP are
only responsible for preparing everything that RTP needs, like
determining the address where RTP packets need to be sent and
negotiating the format that audio and video need to be encoded in
(i.e. codec), etc. They also take care of things like locating users,
maintaining their presence, making the phones ring, and many
others. This is why protocols like SIP and XMPP are often referred to
as signalling protocols.

What does this mean in the context of Jitsi? Well, first of all it
means that you are not going to find any code manipulating audio or
video flows in either the \emph{sip} or \emph{jabber} jitsi packages.
This kind of code lives in our MediaService. The MediaService and its
implementation are located in
\path{net.java.sip.communicator.service.neomedia} and
\path{net.java.sip.communicator.impl.neomedia}.

\begin{aosabox}{Why ``neomedia''?}

The ``neo'' in the neomedia package name indicates that it replaces a
similar package that we used originally and that we then had to
completely rewrite. This is actually how we came up with one of our
rules of thumb: It is hardly ever worth it to spend a lot of time
designing an application to be 100\% future-proof. There is simply no
way of taking everything into account, so you are bound to have to
make changes later anyway. Besides, it is quite likely that a
painstaking design phase will introduce complexities that you will
never need because the scenarios you prepared for never happen.

\end{aosabox}

In addition to the MediaService itself, there are two other interfaces
that are particularly important: MediaDevice and MediaStream.

\begin{aosasect2}{Capture, Streaming, and Playback}

MediaDevices represent the capture and playback devices that we use
during a call (\aosafigref{fig.jit.media}). Your microphone and
speakers, your headset and your webcam are all examples of such
MediaDevices, but they are not the only ones.  Desktop streaming and
sharing calls in Jitsi capture video from your desktop, while a
conference call uses an AudioMixer device in order to mix the audio we
receive from the active participants. In all cases, MediaDevices
represent only a single MediaType. That is, they can only be either
audio or video but never both. This means that if, for example, you
have a webcam with an integrated microphone, Jitsi sees it as two
devices: one that can only capture video, and another one that can
only capture sound.

Devices alone, however, are not enough to make a phone or a video
call.  In addition to playing and capturing media, one has to also be
able to send it over the network. This is where MediaStreams come
in. A MediaStream interface is what connects a MediaDevice to your
interlocutor. It represents incoming and outgoing packets that you
exchange with them within a call.

Just as with devices, one stream can be responsible for only one
MediaType. This means that in the case of an audio/video call Jitsi
has to create two separate media streams and then connect each to the
corresponding audio or video MediaDevice.

\aosafigureTop{../images/jitsi/Media.png}{Media Streams For Different Devices}{fig.jit.media}

\end{aosasect2}

\begin{aosasect2}{Codecs}

Another important concept in media streaming is that of MediaFormats,
also known as codecs. By default most operating systems let you
capture audio in 48KHz PCM or something similar.  This is what we
often refer to as ``raw audio'' and it's the kind of audio you get in
WAV files: great quality and enormous size. It is quite impractical to
try and transport audio over the Internet in the PCM format.

This is what codecs are for: they let you present and transport audio
or video in a variety of different ways.  Some audio codecs like iLBC,
8KHz Speex, or G.729, have low bandwidth requirements but sound
somewhat muffled.  Others like wideband Speex and G.722 give you great
audio quality but also require more bandwidth. There are codecs that
try to deliver good quality while keeping bandwidth requirements at a
reasonable level.  H.264, the popular video codec, is a good example
of that. The trade-off here is the amount of calculation required
during conversion. If you use Jitsi for an H.264 video call you see a
good quality image and your bandwidth requirements are quite
reasonable, but your CPU runs at maximum.

All this is an oversimplification, but the idea is that codec choice
is all about compromises. You either sacrifice bandwidth, quality, CPU
intensity, or some combination of those. People working with VoIP
rarely need to know more about codecs.

\end{aosasect2}

\begin{aosasect2}{Connecting with the Protocol Providers}

Protocols in Jitsi that currently have audio/video support all use our
MediaServices exactly the same way. First they ask the MediaService
about the devices that are available on the system:

\begin{verbatim}
public List<MediaDevice> getDevices(MediaType mediaType, MediaUseCase useCase);
\end{verbatim}

\noindent The MediaType indicates whether we are interested in audio or video
devices. The MediaUseCase parameter is currently only considered in
the case of video devices. It tells the media service whether we'd
like to get devices that could be used in a regular call
(MediaUseCase.CALL), in which case it returns a list of available
webcams, or a desktop sharing session (MediaUseCase.DESKTOP), in which
case it returns references to the user desktops.

The next step is to obtain the list of formats that are available for
a specific device. We do this through the
\code{MediaDevice.getSupportedFormats} method:

\begin{verbatim}
public List<MediaFormat> getSupportedFormats();
\end{verbatim}

\noindent Once it has this list, the protocol implementation sends it to the
remote party, which responds with a subset of them to indicate which
ones it supports. This exchange is also known as the Offer/Answer
Model and it often uses the Session Description Protocol or some form
of it.

After exchanging formats and some port numbers and IP addresses, VoIP
protocols create, configure and start the MediaStreams. Roughly
speaking, this initialization is along the following lines:

\begin{verbatim}
// first create a stream connector telling the media service what sockets
// to use when transport media with RTP and flow control and statistics
// messages with RTCP
StreamConnector connector =  new DefaultStreamConnector(rtpSocket, rtcpSocket);
MediaStream stream = mediaService.createMediaStream(connector, device, control);

// A MediaStreamTarget indicates the address and ports where our
// interlocutor is expecting media. Different VoIP protocols have their
// own ways of exchanging this information
stream.setTarget(target);

// The MediaDirection parameter tells the stream whether it is going to be
// incoming, outgoing or both
stream.setDirection(direction);

// Then we set the stream format. We use the one that came
// first in the list returned in the session negotiation answer.
stream.setFormat(format);

// Finally, we are ready to actually start grabbing media from our
// media device and streaming it over the Internet
stream.start();
\end{verbatim}

\noindent Now you can wave at your webcam, grab the mic and say, ``Hello
world!''

\end{aosasect2}

\end{aosasect1}

\begin{aosasect1}{UI Service}

So far we have covered parts of Jitsi that deal with protocols,
sending and receiving messages and making calls. Above all, however,
Jitsi is an application used by actual people and as such, one of its
most important aspects is its user interface. Most of the time the
user interface uses the services that all the other bundles in Jitsi
expose. There are some cases, however, where things happen the other
way around.

Plugins are the first example that comes to mind. Plugins in Jitsi
often need to be able to interact with the user. This means they have
to open, close, move or add components to existing windows and panels
in the user interface. This is where our UIService comes into play. It
allows for basic control over the main window in Jitsi and this is how
our icons in the Mac OS X dock and the Windows notification area let
users control the application.

In addition to simply playing with the contact list, plugins can also
extend it. The plugin that implements support for chat encryption
(OTR) in Jitsi is a good example for this. Our OTR bundle needs to
register several GUI components in various parts of the user
interface. It adds a padlock button in the chat window and a
sub-section in the right-click menu of all contacts.

The good news is that it can do all this with just a few method calls.
The OSGi activator for the OTR bundle, OtrActivator, contains the
following lines:

\begin{verbatim}
Hashtable<String, String> filter = new Hashtable<String, String>();

// Register the right-click menu item.
filter(Container.CONTAINER_ID,
    Container.CONTAINER_CONTACT_RIGHT_BUTTON_MENU.getID());

bundleContext.registerService(PluginComponent.class.getName(),
    new OtrMetaContactMenu(Container.CONTAINER_CONTACT_RIGHT_BUTTON_MENU),
    filter);

// Register the chat window menu bar item.
filter.put(Container.CONTAINER_ID,
           Container.CONTAINER_CHAT_MENU_BAR.getID());

bundleContext.registerService(PluginComponent.class.getName(),
           new OtrMetaContactMenu(Container.CONTAINER_CHAT_MENU_BAR),
           filter);
\end{verbatim}

As you can see, adding components to our graphical user interface
simply comes down to registering OSGi services. On the other side of
the fence, our UIService implementation is looking for implementations
of its PluginComponent interface. Whenever it detects that a new
implementation has been registered, it obtains a reference to it and
adds it to the container indicated in the OSGi service filter.

Here's how this happens in the case of the right-click menu
item. Within the UI bundle, the class that represents the right click
menu, MetaContactRightButtonMenu, contains the following lines:

\begin{verbatim}
// Search for plugin components registered through the OSGI bundle context.
ServiceReference[] serRefs = null;

String osgiFilter = "("
    + Container.CONTAINER_ID
    + "="+Container.CONTAINER_CONTACT_RIGHT_BUTTON_MENU.getID()+")";

serRefs = GuiActivator.bundleContext.getServiceReferences(
        PluginComponent.class.getName(),
        osgiFilter);
// Go through all the plugins we found and add them to the menu.
for (int i = 0; i < serRefs.length; i ++)
{
    PluginComponent component = (PluginComponent) GuiActivator
        .bundleContext.getService(serRefs[i]);

    component.setCurrentContact(metaContact);

    if (component.getComponent() == null)
        continue;

    this.add((Component)component.getComponent());
}
\end{verbatim}

\noindent And that's all there is to it. Most of the windows that you see within
Jitsi do exactly the same thing: They look through the bundle context
for services implementing the PluginComponent interface that have a
filter indicating that they want to be added to the corresponding
container.  Plugins are like hitch-hikers holding up signs with the
names of their destinations, making Jitsi windows the drivers who pick
them up.

\end{aosasect1}

\begin{aosasect1}{Lessons Learned}

When we started work on SIP Communicator, one of the most common
criticisms or questions we heard was: ``Why are you using Java? Don't
you know it's slow? You'd never be able to get decent quality for
audio/video calls!'' The ``Java is slow'' myth has even been repeated
by potential users as a reason they stick with Skype instead of trying
Jitsi. But the first lesson we've learned from our work on the project
is that efficiency is no more of a concern with Java than it would
have been with C++ or other native alternatives.

We won't pretend that the decision to choose Java was the result of
rigorous analysis of all possible options. We simply wanted an easy
way to build something that ran on Windows and Linux, and Java and the
Java Media Framework seemed to offer one relatively easy way of doing
so.

Throughout the years we haven't had many reasons to regret this
decision.  Quite the contrary: even though it doesn't make it
completely transparent, Java does help portability and 90\% of the code
in SIP Communicator doesn't change from one OS to the next. This
includes all the protocol stack implementations (e.g.,  SIP, XMPP, RTP,
etc.) that are complex enough as they are. Not having to worry about
OS specifics in such parts of the code has proven immensely useful.

Furthermore, Java's popularity has turned out to be very important
when building our community. Contributors are a scarce resource as it
is.  People need to like the nature of the application, they need to
find time and motivation---all of this is hard to muster. Not
requiring them to learn a new language is, therefore, an advantage.

Contrary to most expectations, Java's presumed lack of speed has
rarely been a reason to go native. Most of the time decisions to use
native languages were driven by OS integration and how much access
Java was giving us to OS-specific utilities.  Below we discuss
the three most important areas where Java fell short.

\begin{aosasect2}{Java Sound vs.\ PortAudio}

Java Sound is Java's default API for capturing and playing audio. It
is part of the runtime environment and therefore runs on all the
platforms the Java Virtual Machine comes for. During its first years
as SIP Communicator, Jitsi used JavaSound exclusively and this
presented us with quite a few inconveniences.

First of all, the API did not give us the option of choosing which
audio device to use. This is a big problem. When using their computer
for audio and video calls, users often use advanced USB headsets or
other audio devices to get the best possible quality. When multiple
devices are present on a computer, JavaSound routes all audio through
whichever device the OS considers default, and this is not good enough
in many cases. Many users like to keep all other applications running
on their default sound card so that, for example, they could keep
hearing music through their speakers. What's even more important is
that in many cases it is best for SIP Communicator to send audio
notifications to one device and the actual call audio to another,
allowing a user to hear an incoming call alert on their speakers even
if they are not in front of the computer and then, after picking up
the call, to start using a headset.

None of this is possible with Java Sound. What's more, the Linux
implementation uses OSS which is deprecated on most of today's Linux
distributions.

We decided to use an alternative audio system. We didn't want to
compromise our multi-platform nature and, if possible, we wanted to
avoid having to handle it all by ourselves. This is where 
PortAudio\footnote{\url{http://portaudio.com/}} came in extremely handy.

When Java doesn't let you do something itself, cross-platform open
source projects are the next best thing. Switching to PortAudio has
allowed us to implement support for fine-grained configurable audio
rendering and capture just as we described it above. It also runs on
Windows, Linux, Mac OS X, FreeBSD and others that we haven't had the
time to provide packages for.

\end{aosasect2}

\begin{aosasect2}{Video Capture and Rendering}

Video is just as important to us as audio. However, this didn't seem
to be the case for the creators of Java, because there is no default
API in the JRE that allows capturing or rendering video. For a while
the Java Media Framework seemed to be destined to become such an API
until Sun stopped maintaining it.

Naturally we started looking for a PortAudio-style video alternative,
but this time we weren't so lucky. At first we decided to go with the
LTI-CIVIL framework from Ken
Larson\footnote{\url{http://lti-civil.org/}}. This is a wonderful
project and we used it for quite a while\footnote{Actually we still have it as
a non-default option.}. However it turned out to be suboptimal
when used in a real-time communications context.

So we came to the conclusion that the only way to provide impeccable
video communication for Jitsi would be for us to implement native
grabbers and renderers all by ourselves. This was not an easy decision
since it implied adding a lot of complexity and a substantial
maintenance load to the project but we simply had no choice: we really
wanted to have quality video calls. And now we do!

Our native grabbers and renderers directly use Video4Linux 2, QTKit
and DirectShow/Direct3D on Linux, Mac OS X, and Windows respectively.

\end{aosasect2}

\begin{aosasect2}{Video Encoding and Decoding}

SIP Communicator, and hence Jitsi, supported video calls from its
first days. That's because the Java Media Framework allowed encoding
video using the H.263 codec and a 176x144 (CIF) format.  Those of you
who know what H.263 CIF looks like are probably smiling right now; few
of us would use a video chat application today if that's all it had to
offer.

In order to offer decent quality we've had to use other libraries like
FFmpeg. Video encoding is actually one of the few places where Java
shows its limits performance-wise. So do other languages, as evidenced
by the fact that FFmpeg developers actually use Assembler in a number
of places in order to handle video in the most efficient way possible.

\end{aosasect2}

\begin{aosasect2}{Others}

There are a number of other places where we've decided that we needed
to go native for better results. Systray notifications with Growl on
Mac OS X and libnotify on Linux are one such example. Others include
querying contact databases from Microsoft Outlook and Apple Address
Book, determining source IP address depending on a destination, using
existing codec implementations for Speex and G.722, capturing desktop
screenshots, and translating chars into key codes.

\end{aosasect2}

The important thing is that whenever we needed to choose a native
solution, we could, and we did. This brings us to our point: Ever
since we've started Jitsi we've fixed, added, or even entirely
rewritten various parts of it because we wanted them to look, feel or
perform better.  However, we've never ever regretted any of the things
we didn't get right the first time. When in doubt, we simply picked
one of the available options and went with it. We could have waited
until we knew better what we were doing, but if we had, there
would be no Jitsi today.

\end{aosasect1}

\begin{aosasect1}{Acknowledgments}

Many thanks to Yana Stamcheva for creating all the diagrams in this
chapter.

\end{aosasect1}

\end{aosachapter}
